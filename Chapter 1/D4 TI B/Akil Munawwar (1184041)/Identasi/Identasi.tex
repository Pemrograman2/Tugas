\documentclass{article}
\begin{document}
\title{Identasi}
\author{Akil Munawwar \\ D4 TI 2B \\ 1184041}
\maketitle

\part{Identasi}
\section{Pengertian Identasi}
Identasi adalah tulisan yang menjorok. Dalam Python, identasi sangat mempengaruhi hasil. Hal ini dikarenakan jika koding berada pada satu sisi kiri yang sama maka akan dibaca sebagai satu blok dan juga identasi sangat berpengaruh pada perintah if else dibutuhkan nya identasi untuk memisahkan bagian if dan bagian else.
\section{Jenis Error}
\begin{enumerate}
\item IdentationError expected an intended block
\end{enumerate}
\section{Cara Membaca Error}
\begin{enumerate}
\item Akan ada diberi error tanda kuning disebelah kiri kodingan jika memakai spyder
\item Diberi petunjuk dibagian mana errornya pada bagian console spyder
\end{enumerate}
\section{Cara Menangani Error}
\begin{enumerate}
\item Ketika menjumpai error intended block, maka hal pertama kita perhatikan adalah dibagian mana error nya. Seperti jika dia berada di satu baris, maka yang diperlukan ialah cukup tekan tab untuk memajukan satu baris kedepan agar tidak terjadi lagi error tersebut.
\end{enumerate}
\end{document}