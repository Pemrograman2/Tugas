\chapter*{Teori}

\begin{enumerate}
	\item CSV(Comma Separated Value) adalah format data yang digunakan para pengguna untuk menginputkan data kedalam database secara sederhana.

	\begin{itemize}
	\item Fungsi file CSV adalah format basis data sederhana yang di setiap record dipisahkan dengan tanda koma dan titik koma.
	\end{itemize}
	
	\begin{figure} [h]
	\includegraphics[width=5cm]{poto/1.png}
	\centering
	\end{figure}		
	
	
	
	
	\item Aplikasi yang dapat membuat file CSV adalah Notepad,Worpad,Microsoft Excel,dll
	
	\item Cara membuat file CSV:
	\begin{itemize}
	\item Buka Ms.Excell
	\end{itemize}
	\begin{itemize}
	\item Membuat data di Ms.Excel
	\end{itemize}
	\begin{itemize}
	\item Lalu pada saat menyimpan pilih "save as" dan jenis file nya diganti csv
	\end{itemize}
	\begin{itemize}
	\item Dan file CSV telah dibuat
	\end{itemize}

	\item library csv adalah format yang sudah digunakan selama bertahun-tahun dengan cara standar di RFC 4180. Perbedaan halus terdapat di beberapa aplikasi.
	
	
	\item library Pandas adalah alat analisis data dan struktur unruk bahasa pemrograman Python.  Panda digunakan dengan mudah untuk mengelola data salah satu fiturnya adalah Dataframe. Dataframe dapat digunakan untuk membaca sebuah file dan menjadikannya table.


	\item Fungsi yang terdapat pada library CSV:\\
	\begin{itemize}
	\item Reade, Fungsi Reader diguunakan untuk membaca  isi file
	\end{itemize}
	\begin{itemize}
	\item Dict Reader, Fungsi Dict Reader digunakan untuk membaca isi file yang terdapat di dictionary
	\end{itemize}
	\begin{itemize}
	\item Write, Fungsi Write ini digunakan untuk menulis file
	\end{itemize}
	\begin{itemize}
	\item Dict Write, Fungsi Dict Write digunakan untuk menulis file yang ada di dictionary
	\end{itemize}
	
	\item Fungsi yang terdapat di libarary pandas
	\begin{itemize}
	\item to\_csv, untuk menulis file yang type nya CSV
	\end{itemize}
	\begin{itemize}
	\item read\_csv, untuk membaca file type CSV
	\end{itemize}
	
	
\end{enumerate}


