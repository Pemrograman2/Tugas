\documentclass[a4paper, 12pt]{article}

\usepackage{babel}
\usepackage{enumitem}
\usepackage{times}
\usepackage{graphicx}
\usepackage{geometry}
	\geometry{left = 4cm, top = 4cm, right = 3cm, bottom = 3cm}
\usepackage{float}
\usepackage{setspace}
	\setstretch{1.5}
\usepackage{listings}


\begin{document}
\title{\huge\textbf{Tugas Praktikum Pemrograman II (Chapter 4)}}
\date{}

\maketitle


\begin{figure}[!ht]
\begin{center}
\includegraphics[width = 6cm, height = 6cm]{poltekpos.jpg}
\end{center}
\end{figure}

\begin{center}
\vspace{1cm}
Disusun oleh :\\
Annisa Khairani Febrianti\\
D4 TI 2C\\
1.18.4.071\\
\vspace{1cm}
\textbf{PROGRAM DIPLOMA IV POLITEKNIK POS INDONESIA} \linebreak
\textbf{POLITEKNIK POS INDONESIA} \linebreak
\textbf{BANDUNG}\linebreak
\textbf{2019}

\end{center}


\thispagestyle{empty}

\chapter{Pengelolaan File CSV}

Tujuan pembelajaran pada pertemuan keempat antara lain:
\begin{enumerate}
\item
Mengenal file CSV dan fungsinya 
\item
Mengerti cara memakai library CSV
\item
Mengerti cara memakai library pandas
\item
Mengatasi Error yang terjadi akibat pemakaian library csv dan pandas
\item
Try Except
\end{enumerate}
Tugas dengan cara dikumpulkan dengan pull request ke github dengan menggunakan latex pada repo yang dibuat oleh asisten IRC. Kode program dipisah dalam folder src NPM.py yang berisi praktek dari masing-masing tugas file terpisah sesuai nomor yang kemudian dipanggil menggunakan input listing ke dalam file latex penjelasan atau nomor pengerjaan. Masing masing soal bernilai 5 dengan total nilai 100. Gunakan bahasa yang baku dan bebas plagiat dengan dibuktikan hasil scan plagiarisme. Serta hasil scrinsut dari komputer sendiri, dan kode hasil sendiri. Pengerjaan menggunakan latex dan harus menyertakan file pdf hasil compile pdflatex, jika tidak diskon 50\%.


\section{Pemahaman Teori}
Kerjakan soal berikut ini, masing masing bernilai 5. Untuk hari pertama.
Praktek teori penunjang yang dikerjakan dengan deadline besok jam 4 pagi:
\begin{enumerate}
\item
Apa itu fungsi file csv, jelaskan sejarah dan contoh
\par\textbf{Jawaban}
\par CSV (comma separated value) merupakan sebuah file yang berfungsi untuk mempresentasikan sebuah data. format ini termasuk dalam standar file ASCH. file csv juga dapat diimplementasikan dalam beberapa macam perangkat lunak contohnya ms word,oracle,notepad,MySql,sublime dan lain-lain.
\par \textbf{Contoh :}
\par
      \begin{centering}
          \centering
          \includegraphics[scale=1]{figures/chapter 4/1.PNG}
      \end{centering}
\item
Aplikasi-aplikasi apa saja yang bisa menciptakan file csv?
\par \textbf{Jawaban}
\begin{itemize}
    \item Microsoft excel
    \item Notepad
    \item Sublime
    \item Google Sheet
    \item dll
\end{itemize}
\item
Jelaskan bagaimana cara menulis dan membaca file csv di excel atau spreadsheet
\par\textbf{Jawaban}
\begin{itemize}
    \item Buka program microsoft excel atau spreadsheet
    \item Buat data pada baris dan kolom
    \item Lalu save as dan memilih save file dengan .csv
\end{itemize}
\item
Jelaskan sejarah library csv
\par\textbf{Jawaban} 
\par Liblary csv mengimplementasikan kelas yang digunakan untuk membaca dan menulis data dalam format .csv. format ini termasuk dalam standar file ASCH. file csv juga dapat diimplementasikan dalam beberapa macam perangkat lunak contohnya ms word,oracle,notepad,MySql,sublime dan lain-lain.
\item
Jelaskan sejarah library pandas
\par\textbf{Jawaban}
\par Liblaryu pandas adalah pustaka perangkat lunak yang ditulis untuk bahasa pemrograman Python untuk manipulasi dan analisis data.
\item
Jelaskan fungsi-fungsi yang terdapat di library csv
\par\textbf{Jawaban}
\begin{itemize}
    \item Reader
    \par Reader berfungsi untuk membaca file .csv
    \item Write
    \par Write berfungsi untuk membuat file .csv
    \item DictReader
    \par Dictreader berfungsi membaca file .csv dari dictionary
    \item DictWriter   
    \par Dictreader berfungsi untuk membaca file .csv dari dictionary
\end{itemize}
\item
Jelaskan fungsi-fungsi yang terdapat di library pandas
\begin{itemize}
    \item read .csv
    \par fungsi ini digunakan untuk membaca atau membuka file .csv
    \item to.csv
    \par fungsi ini digunakan untuk membuat file .csv
\end{itemize}
\end{enumerate}

\section{Ketrampilan Pemrograman}
Kerjakan soal berikut ini, masing masing bernilai 5 untuk hari kedua, lusa jam 4 pagi. Soalnya adalah:

\begin{enumerate}
\item
Buatlah fungsi (file terpisah/library dengan nama NPM\_csv.py) untuk membuka file csv dengan lib csv mode list
\item
Buatlah fungsi (file terpisah/library dengan nama NPM\_csv.py) untuk membuka file csv dengan lib csv mode dictionary
\item
Buatlah fungsi (file terpisah/library dengan nama NPM\_pandas.py) untuk membuka file csv dengan lib pandas mode list
\item
Buatlah fungsi (file terpisah/library dengan nama NPM\_pandas.py) untuk membuka file csv dengan lib pandas mode dictionary
\item
Buat fungsi baru di NPM\_pandas.py untuk mengubah format tanggal menjadi standar dataframe
\item
Buat fungsi baru di NPM\_pandas.py untuk mengubah index kolom
\item
Buat fungsi baru di NPM\_pandas.py untuk mengubah atribut atau nama kolom
\item
Buat program main.py yang menggunakan library NPM\_csv.py yang membuat dan membaca file csv
\item
Buat program main2.py yang menggunakan library NPM\_pandas.py yang membuat dan membaca file csv
\end{enumerate}




\section{Ketrampilan Penanganan Error}
Kerjakan soal berikut ini, masing masing bernilai 5(hari kedua). Bagian Penanganan error dari script python.
\begin{enumerate}
\item
Tuliskan peringatan error yang didapat dari mengerjakan praktek ketiga ini, dan jelaskan cara penanganan error tersebut.
dan Buatlah satu fungsi yang menggunakan gunakan try except untuk menanggulangi error tersebut.
\begin{itemize}
    \item  berikut errornya yaitu penggunaan variable yang tidak tepat
      \begin{center}
        \centering
        \includegraphics[scale=1]{figures/chapter 3/14.PNG}
    \end{center}
    
        \item  cara penanganannya
      \begin{center}
        \centering
        \includegraphics[scale=1]{figures/chapter 3/15.PNG}
    \end{center}
    
    \end{itemize}
\end{enumerate}



\section{Presentasi Tugas}
Pada pertemuan ini, diadakan dua penilaiain yaitu penilaian untuk tugas mingguan seperti sebelumnya dengan nilai maksimal 100. Kemudian dalam satu minggu kedepan maksimal sebelum waktu mata kuliah kecerdasan buatan. Ada presentasi kematerian dengan nilai presentasi yang terpisah masing-masing 100. Jadi ada tiga komponen penilaiain pada pertemuan ini yaitu :
\begin{enumerate}
	\item tugas minggu hari ini dan besok (maks 100). pada chapter ini
	\item presentasi csv (maks 100). Mempraktekkan kode python dan menjelaskan cara kerjanya.
\end{enumerate}
Waktu presentasi pada jam kerja di IRC. Kriteria penilaian presentasi sangat sederhana, presenter akan ditanyai 20(10 pertanyaan program, 10 pertanyaan teori) pertanyaan tentang pemahamannya menggunakan python untuk kecerdasan buatan. jika presenter tidak bisa menjawab satu pertanyaan asisten maka nilai nol. Jika semua pertanyaan bisa dijawab maka nilai 100. Presentasi bisa diulang apabila gagal, sampai bisa mendapatkan nilai 100 dalam waktu satu minggu kedepan.





\section{PEMAHAMAN TEORI}
\begin{enumerate}
\item Apa itu fungsi file csv, jelaskan sejarah dan contoh
\subsection{Fungsi file CSV}
File Comma Separated Value (CSV) adalah Format data yang memudahkan penggunaanya untuk memudahkan pengguna melakukan suatu penginputan data ke database secara sederhana dan digunakan untuk bertukar data antara aplikasi yang berbeda. CSV bisa digunakan dalam standar file ASCII yang artinya setiap record dipisahkan dengan tanda(,) atau titik koma(;).Format CSV biasanya digunakan oleh perusahaan besar seperti yayasan, sekolah karena merekalah yang memiliki basis data yang sangat besar dalam penggunaan csv yang memungkinkan pencarian data menjadi lebih mudah dengan menggunakan WordPad.File CSV digunakan untuk menyimpan informasi yang dipisahkan oleh koma, bukan menyimpan informasi dalam kolom.Dan jika teks dan angka disimpan dalam file csv maka mudah untuk memindahkannya dari satu program ke program lain.
\subsection{Sejarah}
	IBM Fortran (level H extended) compiler di bawah OS/360 mendukung fomat CSV pada tahun 1972. FORTRAN 77 mendefinisikan penulisannya dimana input atau output yang menggunakan tanda koma atau spasi untuk pembatas antara data dan penulisan tersebut sudah disetujui pada tahun 1978.
\hfill\break
	Osborne Executive computer yang mengembangkan SuperCalc spreadsheet pada tahun 1983 membuat konvensi kutipan CSV yang memungkinkan string mengandung koma(,).
\hfill\break
	Inisiatif standardisasi utama \- mentransformasi \"definisi fuzzy defacto\" menjadi definisi yang lebih tepat dan de jure \-adalah pada tahun 2005, dengan RFC4180, mendefinisikan CSV sebagai Tipe Konten MIME. Kemudian, pada tahun 2013, beberapa kekurangan RFC4180 ditangani oleh rekomendasi W3C.
\hfill\break
	Pada 2014 IETF menerbitkan RFC7111 yang menjelaskan aplikasi fragmen URI pada dokumen CSV. RFC7111 menentukan bagaimana rentang baris, kolom, dan sel dapat dipilih dari dokumen CSV menggunakan indeks posisi.
\hfill\break
	Pada 2015 W3C, dalam upaya meningkatkan CSV dengan semantik formal, mempublikasikan draftrekomendasi pertama untuk stadar metadata CSV, yang dimulai sebagai rekomendasi pada bulan Desembertahun yang sama.
\subsection{contoh}
\lstinputlisting[caption = Contoh penggunaan format CSV., firstline=1, lastline=3]{figures/src/Teori1.csv}
\item Aplikasi-aplikasi apa saja yang bisa menciptakan suatu file dalam bentuk csv
\begin{enumerate}
\item Editor Text (Sublime, Notepad, Atom, visual studio code dan lain-lain) 
\item Spreadshett (Microsoft Excel, google spreadsheet,libreOfficecalc dan lain-lain)
\end{enumerate}
\item Jelaskan bagaimana cara menulis dan membaca file csv di excel atau spreadsheet
\begin{enumerate}
\subsection{Menulis File CSV}
\item Pertama-tama kita buka terlebih dahulu aplikasi microsoft excel, dengan cara klik start dan cari excel kemudian enter
	\begin{figure}[H]
			\includegraphics[width=8cm]{figures/1.png}
			\centering
			\caption{Kemudian kita Klik start lalu cari excel}
	\end{figure} 
\item Kemudian setelah aplikasi microsoft excel terbuka lalu kita klik Blank Workbokk
 	\begin{figure}[H]
			\includegraphics[width=8cm]{figures/2.png}
			\centering
			\caption{Kemudian kita Klik kembali Blank Workbook}
	\end{figure} 
\item Setelah Blank Workbook terbuka lalu kita tulis sesuai data yang diinginkan 
 	\begin{figure}[H]
			\includegraphics[width=8cm]{figures/3.png}
			\centering
			\caption{Kemudian kita tulis sesuai data yang diinginkan}
	\end{figure} 
\item Setelah data selesai dibuat maka save file tersebut dengan cara klik file lalu save as dan piih browse
 	\begin{figure}[H]
			\includegraphics[width=8cm]{figures/4.png}
			\centering
			\caption{Setelah itu kita Pilih save as kemudian kita pilih browse}
	\end{figure} 
\item Lalu kita beri nama data filenya pada File Name dan ubah type file pada Save as type menjadi .csv
 	\begin{figure}[H]
			\includegraphics[width=8cm]{figures/5.png}
			\centering
			\caption{Lalu kita beri nama file dan ubah save as type nya}
	\end{figure}
\item Kemudian kita klik save
 	\begin{figure}[H]
			\includegraphics[width=8cm]{figures/6.png}
			\centering
			\caption{Klik save}
	\end{figure} 
\item Kemudian file yang telah dibuat tadi tersimpan dengan ekstensi .csv. Dan untuk melihat isi filenya tinggal klik dua kali pada file tersebut.
	\begin{figure}[H]
			\includegraphics[width=8cm]{figures/7.png}
			\centering
			\caption{Setelah itu data berhasil dibuat}
	\end{figure}
\item Ini adalah isi file yang telah dibuat
	\begin{figure}[H]
			\includegraphics[width=8cm]{figures/8.png}
			\centering
			\caption{isi file yang telah dibuat}
	\end{figure}
\subsection{Untuk melihat File CSV di Excel atau Spreadsheet}
\item Pertama kita klik dua kali pada file yang yang berekstensi CSV.
	\begin{figure}[H]
			\includegraphics[width=8cm]{figures/9.png}
			\centering
			\caption{Kemudian kita Klik dua kali file berekstensi .csv}
	\end{figure}
\item Setelah itu file akan terbuka secara otomatis di aplikasi Excel atau spreadsheet.
	\begin{figure}[H]
			\includegraphics[width=8cm]{figures/10a.png}
			\centering
			\caption{Isi data yang telah dibuat}
	\end{figure}
\end{enumerate} 
\item Sejarah library csv
\hfill\break
Library csv mengimplementasikan kelas yang digunakan untuk membaca dan menulis data dalam format csv. Ini memungkinkan seorang programmer untuk mengatakan "baca data dari file ini yang dihasilkan oleh Excel". Pemrogram juga bisa menentukan format csv sesuai dengan keinginan mereka sendiri.
\item sejarah library pandas
\hfill\break
Pengembangan pandas dimulai pada tahun 2008 di AQR Captal Management. Pandas pada akhir 2009 telah menjadi open source dan secara aktif didukung oleh komunitas individu yang berpikiran sama dengan seluruh indonesia yang menyumbangkan waktu dan energi mereka untuk membuat sebuah pandas yang bersifat open source. 
\hfill\break
Pandas adalah sebuah proyek yang di sponsori oleh NumFOCUS sejak 2005. Ini akan sangat membantu untuk memastikan sebuah keberhasilan pengembangan pandas sebagai sebuah proyek sumber terbuka kelas dunia.
\item Jelaskan fungsi-fungsi yang terdapat di library csv
\begin{enumerate}
\item reader
\hfill\break
	Reader memiliki fungsi yang digunakan untuk membaca isi file yang berformat CSV dari list.
	\lstinputlisting[caption = Membaca file berformat CSV list., firstline=1,lastline=14]{figures/src/Teori1.py}
\item DictReader
\hfill\break
	DictReader memiliki fungsi yang digunakan untuk membaca isi file yang berformat CSV dari dictionary.
	\lstinputlisting[caption = Membaca file berformat CSV dict., firstline=1,lastline=14]{figures/src/Teori.py}
\item write
\hfill\break
	Write memiliki fungsi yang digunakan untuk menulis pada suatu file yang berformat CSV dari list.
	\lstinputlisting[caption =  Menulis file berformat CSV list., firstline=1, lastline=17]{figures/src/Teori2.py}
\item DictWrite
\hfill\break
	DictWrite juga memiliki fungsi yang digunakan untuk menulis file yang berformat CSV dari dictionary.
	\lstinputlisting[caption =  Menulis file berformat CSV dictionary., firstline=1, lastline=17]{figures/src/Teori3.py}
\end{enumerate}
\item Jelaskan fungsi-fungsi yang terdapat di library pandas
\begin{enumerate}
\item read\_csv
\hfill\break
	Fungsi ini digunakan untuk membaca isi file berformat CSV
	\lstinputlisting[caption =  Membaca file berformat CSV pandas., firstline=1, lastline=12]{figures/src/Teori4.py}
\item to\_csv
\hfill\break
	Fungsi ini digunakan untuk menulis file berformat CSV
	\lstinputlisting[caption =  Menulis file berformat CSV pandas., firstline=1, lastline=13]{figures/src/Teori6.py}
\end{enumerate}
\section{Bukti bebas Plagiarism}
\begin{figure}[H]
			\includegraphics[width=8cm]{figures/ss1.png}
			\centering
			\caption{Bukti Screenshot bebas plagiarism}
	\end{figure}
\section{Ketrampilan Pemrogaman}
\begin{enumerate}
\item Buatlah fungsi (file terpisah/library dengan NPM\_csv.py) untuk membuka file csv dengan lib csv mode list.
\lstinputlisting[caption =  Membuka Mode List Csv., firstline=1, lastline=13]{figures/src/1184071_csv.py}
\item Buatlah fungsi (file terpisah/library dengan nama NPM\_csv.py) untuk membuka file csv dengan lib csv mode dictionary
\lstinputlisting[caption =  Membuka Mode Dict Csv., firstline=15, lastline=20]{figures/src/1184071_csv.py}
\item Buatlah fungsi (file terpisah/library dengan nama NPM\_pandas.py) untuk membuka file csv dengan lib pandas mode list
\lstinputlisting[caption =  Membuka Mode List Pandas., firstline=1, lastline=13]{figures/src/1184071_pandas.py}
\item Buatlah fungsi (file terpisah/library dengan nama NPM\_pandas.py) untuk membuka file csv dengan lib pandas mode dictionary
\lstinputlisting[caption =  Membuka Mode Dict Pandas., firstline=15, lastline=20]{figures/src/1184071_pandas.py}
\item Buat fungsi baru di NPM\_pandas.py untuk mengubah format tanggal menjadi standar dataframe
\lstinputlisting[caption =  Merubah format Tanggal., firstline=21, lastline=24]{figures/src/1184071_pandas.py}
\item Buat fungsi baru di NPM\_pandas.py untuk mengubah index kolom
\lstinputlisting[caption =  Mengubah Index Kolom., firstline=26, lastline=30]{figures/src/1184071_pandas.py}
\item Buat fungsi baru di NPM\_pandas.py untuk mengubah atribut atau nama kolom
\lstinputlisting[caption =  Merubah Nama Kolom., firstline=32, lastline=36]{figures/src/1184071_pandas.py}
\item Buat program main.py yang menggunakan library NPM\_csv.py yang membuat dan membaca file csv
\lstinputlisting[caption =  Membuat dan membaca file., firstline=1, lastline=13]{figures/src/main.py}
\begin{figure}[H]
			\includegraphics[width=8cm]{figures/kp1.png}
			\centering
			\caption{Kode Program Main.py}
	\end{figure}
\item Buat program main2.py yang menggunakan library NPM\_pandas.py yang membuat dan membaca file csv
\lstinputlisting[caption =  Membuat dan membaca file., firstline=1, lastline=13]{figures/src/main2.py}
\begin{figure}[H]
			\includegraphics[width=8cm]{figures/kp2.png}
			\centering
			\caption{Kode Program Main2.py}
	\end{figure}
\end{enumerate}
\section{Bukti bebas Plagiarism}
\begin{figure}[H]
			\includegraphics[width=8cm]{figures/ss2.png}
			\centering
			\caption{Bukti Screenshot bebas plagiarism}
	\end{figure}
\end{enumerate}
\section{Ketrampilan Penanganan Error}
\lstinputlisting[caption =  Fungsi Try Except., firstline=56, lastline=70]{figures/src/1184071.py}
\section{Presentasi Tugas}

\end{document}
