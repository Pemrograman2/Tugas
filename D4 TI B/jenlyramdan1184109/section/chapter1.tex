\chapter{Bab 1}

\section{Sejarah Python}

	Python awal mula dikembangkan oleh Guido van Rossum (1990) di Stichting Mathematisch Centrum (CWI), Amsterdam. Python adalah kelanjutan dari bahasa pemrograman ABC. Pada tahun 1995 Guido pindah ke CNRI di Virginia Amerika sambil terus melanjutkan pengembangan Python. Versi terakhir yang dikeluarkan adalah 1.6. Pada tahun 2000, Guido dan para pengembang inti Python pindah ke BeOpen.com yang merupakan sebuah perusahaan komersial dan membentuk BeOpen PythonLabs. Python 2.0 dikeluarkan oleh BeOpen. Setelah mengeluarkan Python 2.0, Guido dan beberapa anggota tim PythonLabs pindah ke DigitalCreations.
 
	Saat ini pengembangan Python terus dilakukan oleh sekumpulan pemrogram yang dikoordinir Guido dan Python Software Foundation. Python Software Foundation adalah sebuah organisasi non-profit yang dibentuk sebagai pemegang hak cipta intelektual Python sejak versi 2.1 dan dengan demikian mencegah Python dimiliki oleh perusahaan komersial. Saat ini distribusi Python sudah mencapai versi 2.7.14 dan versi 3.6.3 
Nama Python dipilih oleh Guido sebagai nama bahasa ciptaannya karena kecintaan Guido pada acara televisi Monty Python's Flying Circus. Oleh karena itu seringkali ungkapan-ungkapan khas dari acara tersebut seringkali muncul dalam korespondensi antar pengguna Python.

	Python memiliki 2 versi yaitu, versi 2 dan 3. python versi 2 bisa memakai tanda kurung atau tidak.. Namun, di python 3 kita wajib menggunakan tanda kurung. Jika tidak, maka kita akan mendapatkan hasil eror. Dan yang kedua, untuk men cetak 2 buah teks dalam satu baris. Bisa kita lihat sendiri ya perbedaannya di atas.


\section{Impelentasi dan penggunaan python di dunia kerja}

	Python digunakan untuk membuat kecerdasan buatan. website, menganalisa data sampai kecerdasan buatan semuanya memungkinkan dengan python. Contoh implementasi nya suatu perusahaan ingin setiap dalam kurun waktu mengklik tombol. maka dengan kasus berikut dibuat lah timer untuk mengklik tombol ketika kondisi sudah terpenuhi. Berikut perusahaan besar yang menggunakan python yaitu Instagram, Industrial Light and Magic, Netflix dan Google