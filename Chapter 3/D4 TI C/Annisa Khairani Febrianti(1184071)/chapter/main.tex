\documentclass[a4paper, 12pt]{article}

\usepackage{babel}
\usepackage{enumitem}
\usepackage{times}
\usepackage{graphicx}
\usepackage{geometry}
	\geometry{left = 4cm, top = 4cm, right = 3cm, bottom = 3cm}
\usepackage{float}
\usepackage{setspace}
	\setstretch{1.5}
\usepackage{listings}


\begin{document}
\title{\huge\textbf{Tugas Praktikum Pemrograman II (Chapter 3)}}
\date{}

\maketitle


\begin{figure}[!ht]
\begin{center}
\includegraphics[width = 6cm, height = 6cm]{poltekpos.jpg}
\end{center}
\end{figure}

\begin{center}
\vspace{1cm}
Disusun oleh :\\
Annisa Khairani Febrianti\\
D4 TI 2C\\
1.18.4.071\\
\vspace{1cm}
\textbf{PROGRAM DIPLOMA IV POLITEKNIK POS INDONESIA} \linebreak
\textbf{POLITEKNIK POS INDONESIA} \linebreak
\textbf{BANDUNG}\linebreak
\textbf{2019}

\end{center}


\thispagestyle{empty}

\chapter*{instalasi}


\section*{\textit{ Instalasi Python 3}}
\begin{enumerate}
		\item Clik apk anaconda lau clik install. Selanjutnya clik next.
		\begin{figure}[h]
			\includegraphics[width=4cm]{figure/1.png}
			\centering
			\caption{install anaconda}
			\end{figure}
		\item Selanjutnya , clik I agree.
			\begin{figure}[h]
			\includegraphics[width=4cm]{figure/2.png}
			\centering
			\caption{licence agreement}
			\end{figure}
		\item Pilih Just me.
			\begin{figure}[h]
			\includegraphics[width=4cm]{figure/3.png}
			\centering
			\caption{installation type}
			\end{figure}
		\item Pilih lokasi penyimpanan yang akan diinstal, lalu clik next.
			\begin{figure}[h]
			\includegraphics[width=4cm]{figure/4.png}
			\centering
			\caption{lokasi penyimpanan file anaconda}
			\end{figure}
		\item Ceklis bagian ADD Environtment to the Path, hal ini memungkinkan untuk menambahkan environtment anaconda ke dalam path yang ada dalam PC anda. Setelah itu klik next.
			\begin{figure}[h]
			\includegraphics[width=4cm]{figure/5.png}
			\centering
			\caption{menambahkan path environtment}
			\end{figure}
		\item Tunggu instalan sampai selesai.
			\begin{figure}[h]
			\includegraphics[width=4cm]{figure/7.png}
			\centering
			\caption{proses instalasi}
			\end{figure}
		\item Setelah Instal selesai, lau clik next sampai proses terakhir dan clik finish di akhir proses instal.
			\begin{figure}[h]
			\includegraphics[width=4cm]{figure/8.png}
			\centering
			\caption{instalasi selesai}
			\end{figure}
		\item Click next.	
			\begin{figure}[h]
			\includegraphics[width=4cm]{figure/9.png}
			\centering
			\caption{instalasi selesai 2}
			\end{figure}
		\item Click next lagi.	
			\begin{figure}[h]
			\includegraphics[width=4cm]{figure/10.png}
			\centering
			\caption{instalasi selesai 3}
			\end{figure}
		\item Lalu click finish.	
	\end{enumerate}

\section*{\textit{ Instalasi Pip }}
\begin{enumerate}
\item Buka Cmd (CommandPrompt)
\begin{figure}[h]
\includegraphics[width=4cm]{figure/pip1.png}
\centering
\caption{Buka Cmd}
\end{figure}
\item masuk kedalam direktori yang ada file pip nya.
\begin{figure}[h]
\includegraphics[width=4cm]{figure/pip2.png}
\centering
\caption{Masuk kedalam Direktori}
\end{figure}
\item Ketikan syntax untuk install pip nya
\begin{figure}[h]
\includegraphics[width=4cm]{figure/pip3.png}
\centering
\caption{Mengetik Syntax}
\end{figure}
\item Jika sudah ada bacaan successfuly instaled, proses instalasi sudah selesai
\begin{figure}[h]
\includegraphics[width=4cm]{figure/pip4.png}
\centering
\caption{Selesai Install}
\end{figure}
\end{enumerate}

\section*{\textit{ Cara setting environment }}

\begin{enumerate}

\item Buka file exploler, lal click kanan di this pc, pilih properties
\begin{figure}[h]
\centering
\includegraphics[scale=0.8]{figure/setting1.png}
\caption{membuka file exploler}
\end{figure}

\item Buka advance system settings
\begin{figure}[hb]
\centering
\includegraphics[scale=0.8]{figure/setting2.png}
\caption{membuka advance system}
\end{figure}

\item Click environment variables
\begin{figure}[hb]
\includegraphics[width=4cm]{figure/setting3.png}
\centering
\caption{membuka environtment variables}
\end{figure}

\item Jika ingin mensetting pilih editt
\begin{figure}[h]
\includegraphics[width=4cm]{figure/setting4.png}
\centering
\caption{mensetting}
\end{figure}

\section*{\textit{ mencoba entrepreter/cli melakui terminal atau cmd windows }}

\begin{enumerate}

\item Buka Cmd
\begin{figure}[hb]
\includegraphics[width=4cm]{figure/interpreter1.png}
\centering
\caption{membuka Cmd}
\end{figure}

\item Menuliskan syntax
\begin{figure}[hb]
\includegraphics[width=4cm]{figure/interpreter2.png}
\centering
\caption{menuliskan syntax python}
\end{figure}

\item Menuliskan syntax
\begin{figure}[hb]
\includegraphics[width=4cm]{figure/setting3.png}
\centering
\caption{menuliskan syntax}
\end{figure}

\item melihat hasil syntax
\begin{figure}[hb]
\includegraphics[width=4cm]{figure/setting4.png}
\centering
\caption{melihat hasil}
\end{figure}

\end{enumerate}

\section*{\textit{ Menjalankan dan mengupdate anaconda dan spyder }}

\begin{enumerate}

\item Buka aplikasi anaconda, lalu pilih environment.
\begin{figure}[hb]
\includegraphics[width=4cm]{figure/update1.png}
\centering
\caption{membuka aplikasi}
\end{figure}

\item Menuliskan spyder di search package
\begin{figure}[hb]
\includegraphics[width=4cm]{figure/update2.png}
\centering
\caption{search package}
\end{figure}

\item Click kanan pada tanda centang pada bacaan spyder, lalu pilih mark for specific version instalation.
\begin{figure}[hb]
\includegraphics[width=4cm]{figure/update3.png}
\centering
\caption{memilih versi yang akan di install}
\end{figure}

\item Sama seperti yang tadi, search package yang akan di install lalu pilih versi yang akan di up.
\begin{figure}[hb]
\includegraphics[width=4cm]{figure/update5.png}
\centering
\caption{mengupdate anaconda}
\end{figure}

\end{enumerate}

\section*{\textit{ Cara menjalankan Script hello word di spyder }}

\begin{enumerate}

\item Buka aplikasi spyder.
\begin{figure}[hb]
\includegraphics[width=4cm]{figure/pemakaian1.png}
\centering
\caption{membuka aplikasi}
\end{figure}

\item Menuliskan syntax 
\begin{figure}[hb]
\includegraphics[width=4cm]{figure/pemakaian2.png}
\centering
\caption{Menuliskan syntax}
\end{figure}

\item Hasil syntax.
\begin{figure}[hb]
\includegraphics[width=4cm]{figure/pemakaian3.png}
\centering
\caption{Hasil syntax}
\end{figure}

\end{enumerate}

\section*{\textit{Cara menjalankan Script otomatis login aplikasi akademik dengan library selenium dan inputan user}}

\begin{enumerate}

\item Buka aplikasi spyder.
\begin{figure}[hb]
\includegraphics[width=4cm]{figure/pemakaian1.png}
\centering
\caption{membuka aplikasi}
\end{figure}

\item Menuliskan syntax 
\begin{figure}[hb]
\includegraphics[width=4cm]{figure/login.png}
\centering
\caption{Menuliskan syntax}
\end{figure}

\item Hasil syntax.
\begin{figure}[hb]
\includegraphics[width=4cm]{figure/login1.png}
\centering
\caption{Hasil syntax}
\end{figure}

\end{enumerate}


\section*{\textit{ Cara menjalankan Script otomatis login aplikasi akademik dengan library sele-
nium dan inputan user }}

\begin{enumerate}

\item Buka aplikasi spyder.
\begin{figure}[hb]
\includegraphics[width=4cm]{figure/pemakaian1.png}
\centering
\caption{membuka aplikasi}
\end{figure}

\item Menuliskan syntax 
\begin{figure}[hb]
\includegraphics[width=4cm]{figure/login.png}
\centering
\caption{Menuliskan syntax}
\end{figure}

\item Hasil syntax.
\begin{figure}[hb]
\includegraphics[width=4cm]{figure/login1.png}
\centering
\caption{Hasil syntax}
\end{figure}

\end{enumerate}

\section*{\textit{ Cara pemakaian variable explorer di spyder }}

\begin{enumerate}

\item Buka aplikasi spyder.
\begin{figure}[hb]
\includegraphics[width=4cm]{figure/pemakaian1.png}
\centering
\caption{membuka aplikasi}
\end{figure}

\item Menuliskan syntax 
\begin{figure}[hb]
\includegraphics[width=4cm]{figure/pemakaian4.png}
\centering
\caption{Menuliskan syntax}
\end{figure}

\item Hasil syntax.menunjukan nama variable, type, dan value
\begin{figure}[hb]
\includegraphics[width=4cm]{figure/pemakaian5.png}
\centering
\caption{Hasil syntax}
\end{figure}

\end{enumerate}





\section{Fungsi}
	\subsection{pemahaman teori}
		\begin{enumerate}
			\item fungsi\\
			Fungsi adalah blok kode yang akan dieksekusi ketika dipanggil dalam sebuah program.\\
			
			\item Parameter\\
			Parameter adalah inputan sebuah fungsi yang bertujuan untuk menyimpan suatu nilai.
			
			\item Return\\
			Return berfungsi untuk mengembalikan suatu nilai yang telah di proses dalam suatu fungsi dan mengakhiri sebuah fungsi.
			\begin{lstlisting}[language=Python]
			def fungsi(x,y):
					z=x+y
					return z
			\end{lstlisting}
			
			\item Item Paket\\
			Item Paket adalah sebuah direktori dengan suatu file python dan file dengan nama \_init\_.py. jadi sebuah direktori yang didalamnya ada sebuah python dengan nama \_init\_.py,dan dianggap sebagai paket oleh python tersebut, untuk memanggil sebuah paket atau library adalah dengan cara menekan \textit{import} nama paket atau library tersebut lalu paket atau library tersebut dapat digunakan.
			\begin{lstlisting}[language=Python]
			from kampus import mahasiswa
			\end{lstlisting}
			
			\item Class\\
			Class adalah sebuah blueprint dari suatu objek yang akan di bangun.
			\begin{lstlisting}[language=Python]
class World:
    def __init__(self,World):
        self.World = World
    def heloWorld(self):
        print("Helo",World)
			\end{lstlisting}
			
			\item Objek memiliki suatu variabel dan kode yang saling terhubung. objek di buat dengan adanaya suatu class.
			\begin{lstlisting}[language=Python]
#import kelas terlebih dahulu
import kelas3lib
#membuat object
cobakelas=kelas3lib.Kelas3ngitung(npm) 
hasilkelas=cobakelas.npm1()
			\end{lstlisting}
			
			\item Attribut\\
			Attribut adalah sebuah tempat tampungan dari sebuah data atau perintah yang berhubungan dengan attribut tersebut.
			\begin{lstlisting}[language=Python]
Class Kelas3ngitung:
	#pendefinisian attribute
    def __init__(self,World):
        self.World = World
			\end{lstlisting}
			
			\item Method\\
			Method adalah sebuah fungsi yang ada didalam suatu class.
			\begin{lstlisting}[language=Python]
class world:
    def __init__(self,world):
        self.world = world
    #Pembuatan method pada class
    def world(self):
       	print("hello",world,",apa kabar ?")
			\end{lstlisting}
			
			\newpage \item Contoh membuat sebuah library, contoh disini kita membuat pada folder library:
			\begin{lstlisting}{language=Python}
def hello():
    print("Hello world")
			\end{lstlisting}

			\item Contoh jika kita ingin memanggil sebuah fungsi dari suatu library pada main program kita harus terlebih dahulu melakukan import:
			\begin{lstlisting}{language=Python}
#import library yang telah dibuat
import library
#pemanggilan fungsi pada library
library.hello()
			\end{lstlisting}
			
			\item Pemakaian package from kalkulator import penambahan
			\begin{lstlisting}{language=Python}
from kalkulator import penambahan
			\end{lstlisting}
kode diatas merupakan perintah program yang memanggil sebuah package terlebih dahulu baru menambahkan source code penambahan, kode diatas dapat dibaca seperti ini "import penambahan dari folder kalkulator"

			\item Pemanggilan library dalam sebuah folder\\
			untuk mengakses sebuah library dalam sebuah folder kita perlu menuliskan foldernya terlebih dahulu, setelah itu kita mengimport nama librarynya, contoh:
			\begin{lstlisting}{language=Python}
from mahasiswa import kampus
			\end{lstlisting}
artinya dalam package mahasiswa akan memakai atau menggunakan suatu library kampus

			\item Pemanggilan class dalam sebuah folder\\
			untuk mengakses sebuah class dalam sebuah folder kita perlu menuliskan foldernya terlebih dahulu lalu kita  mengimport nama class nya, contoh :
			\begin{lstlisting}{language=Python}
from mahasiswa import kampus
			\end{lstlisting}
artinya dalam package mahasiswa akan memakai atau menggunakan suatu class kampus
			
			\end{enumerate}

    \newpage			
	\subsection{Ketrampilan Pemrograman}
			\begin{enumerate}
				\item \lstinputlisting[language=Python]{src/npm1.py}
				\newpage \item \lstinputlisting[language=Python]{src/npm2.py}
				\item \lstinputlisting[language=Python]{src/npm3.py}
				\item \lstinputlisting[language=Python]{src/npm4.py}
				\item \lstinputlisting[language=Python]{src/npm5.py}
				\item \lstinputlisting[language=Python]{src/npm6.py}
				\item \lstinputlisting[language=Python]{src/npm7.py}
				\item \lstinputlisting[language=Python]{src/npm8.py}
				\item \lstinputlisting[language=Python]{src/npm9.py}
				\item \lstinputlisting[language=Python]{src/npm10.py}
				\item \lstinputlisting[language=Python]{src/lib3.py}
				\item \lstinputlisting[language=Python]{src/kelas3lib.py}
				\item \lstinputlisting[language=Python]{src/main.py}
	
\end{enumerate}

\end{document}

