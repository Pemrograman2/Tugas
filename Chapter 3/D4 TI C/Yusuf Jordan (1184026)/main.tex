\documentclass[12pt]{ociamthesis}  % default square logo 
%\documentclass[12pt,beltcrest]{ociamthesis} % use old belt crest logo
%\documentclass[12pt,shieldcrest]{ociamthesis} % use older shield crest logo

%load any additional packages
\usepackage{amssymb}
\usepackage{listings}
\usepackage{float}
\usepackage{graphics}
\usepackage{color}
 
\definecolor{codegreen}{rgb}{0,0.6,0}
\definecolor{codegray}{rgb}{0.5,0.5,0.5}
\definecolor{codepurple}{rgb}{0.58,0,0.82}
\definecolor{backcolour}{rgb}{0.95,0.95,0.92}
 
\lstdefinestyle{mystyle}{
    backgroundcolor=\color{backcolour},   
    commentstyle=\color{codegreen},
    keywordstyle=\color{magenta},
    numberstyle=\tiny\color{codegray},
    stringstyle=\color{codepurple},
    basicstyle=\footnotesize,
    breakatwhitespace=false,         
    breaklines=true,                 
    captionpos=b,                    
    keepspaces=true,                 
    numbers=left,                    
    numbersep=5pt,                  
    showspaces=false,                
    showstringspaces=false,
    showtabs=false,                  
    tabsize=2,
    language=python
}
 
\lstset{style=mystyle}

%input macros (i.e. write your own macros file called mymacros.tex 
%and uncomment the next line)
%\include{mymacros}

\title{Tugas Chapter 3\\[1ex]     %your thesis title,
        Pemrograman II}   %note \\[1ex] is a line break in the title

\author{Yusuf Jordan El Anwar}             %your name
\college{1184026\\[5ex]
Applied Bachelor of Informatics Engineering}  %your college

%\renewcommand{\submittedtext}{change the default text here if needed}
\degree{Politeknik Pos Indonesia}     %the degree
\degreedate{Bandung 2019}         %the degree date

%end the preamble and start the document
\begin{document}

%this baselineskip gives sufficient line spacing for an examiner to easily
%markup the thesis with comments
\baselineskip=18pt plus1pt

%set the number of sectioning levels that get number and appear in the contents
\setcounter{secnumdepth}{3}
\setcounter{tocdepth}{3}


\maketitle                  % create a title page from the preamble info
\begin{dedication}

”Barang siapa yang menghendaki kehidupan dunia maka wajib baginya memiliki ilmu, \\
dan barang siapa yang menghendaki kehidupan Akherat, \\ 
maka wajib baginya memiliki ilmu, \\
dan barang siapa menghendaki keduanya maka wajib baginya memiliki ilmu”. \\
(HR. Turmudzi) \\

\end{dedication}        % include a dedication.tex file

\begin{romanpages}          % start roman page numbering
				
\end{romanpages}            % end roman page numbering



%now include the files of latex for each of the chapters etc
\chapter{Teori, Sejarah, \& Instalasi Python}
\section{Teori}
Python adalah bahasa pemrograman interpretatif multiguna, python lebih menekankan pada keterbacaan kode agar lebih mudah untuk memahami sintaks.
\section{Sejarah Python}
Python diciptakan oleh Guido van Rossum pertama kali di centrum wiskunde \& informatica (CWI) di Belanda pada awal tahun 1990-an. bahasa yang terinpirasi dari bahasa pemograman ABC, hingga sampai saat ini Guido van Rossum menjadi penulis utama untuk phyton.
Tahun 1995 masih melanjutkan pembuatan phyton di Corporation for National Research Initiative (CNRI) di Virginia Amerika yang meriliskan beberapa bahasa phyton. Diantaranya :
\begin{enumerate}
	\item Python 1.0
	\\ Diliris pada januari tahun 1994
	\item Python 2.0
	\\ Diliris pada 16 Oktober tahun 2000
	\item Python 3.0 
	\\ Diliris pada 3 Desember tahun 2008
\end{enumerate}  
\subsubsection{Penggunaan Python di Perusahaan}
Salah satu bahasa yang banyak dipakai dalam sebuah perusahaan hingga saat ini yaitu bahasa pemograman python,  contoh penggunaan dalam perusahaan yaitu : 
\begin{enumerate}
\item Facebook
\\ Menggunakan framework python "Tornado" yang digunakan untuk menampilkan timeline
\item Instagram 
\\ Menggunakan framework python "Django" yang digunakan sebagai mesin pengelola sisi server dari aplikasi
\item Rasberry pi 
\\ Merupakan perangkat komputer mini yang digunakan sebagi mikrokontroler, bahasa yang digunakannya adalah python
\item NASA
\\ Badan antariksa Amerika ini menggunakan Python untuk bidang sainsnya.
\end{enumerate}
\subsubsection{Perbedaan Python 2 dan 3}
	Python 2 dipublikasikan pada akhir tahun 2000, dinilai lebih transparan dan inklusif untuk pengembangan software ketimbang versi sebelumnya. didukung dengan adanya PEP – Python Enhancement Proposal, dan dilengkapi dengan berbagai fitur programatikal seperti cycle-detecting garbage collector untuk mengotomasi manajemen memori. 
\\ Python 3 merupakan versi yang saat  ini dibuat masih aktif, versi ini banyak perubahan yang dirilis akhir tahun 2008. Fokus dari Python 3 itu sendiri adalah untuk melakukan perapian pada codebase dan menghapuskan duplikasi (redundancy). Python 3 mengalami hambatan pada pengadopsiannya, yang mengakibatkan tidak adanya backwards compatibility dengan Python 2.
\\	 
perbedaan yang mencolok terletak pada : 
\begin{enumerate}
\item Syntak
\item Pembagian pada integer
\end{enumerate}

\section{Instalasi}
\subsection{Instalasi Anaconda}
Berikut ini merupakan tutorial cara menginstalasi Anaconda, yang telah di download di www.anaconda.com setelah itu ikuti langkah-langkah dibawah ini.

	\begin{figure}
	\includegraphics[scale=0.5]{section/1.png}
	\centering
	\caption{Tahap Instalasi 1}
	\end{figure}
	
	\begin{figure}
	\includegraphics[scale=0.5]{section/2.png}
	\centering
	\caption{Tahapan Instalasi 2}
	\end{figure}

	\begin{figure}
	\includegraphics[scale=0.5]{section/3.png}
	\centering
	\caption{Tahapan Instalasi 3}
	\end{figure}

	\begin{figure}
	\includegraphics[scale=0.5]{section/4.png}
	\centering
	\caption{Tahapan Instalasi 4}
	\end{figure}

	\begin{figure}
	\includegraphics[scale=0.5]{section/5.png}
	\centering
	\caption{Tahapan Instalasi 5}
	\end{figure}

	\begin{figure}
	\includegraphics[scale=0.5]{section/6.png}
	\centering
	\caption{Tahapan Instalasi 6}
	\end{figure}

	\begin{figure}
	\includegraphics[scale=0.5]{section/7.png}
	\centering
	\caption{Tahapan Instalasi 7}
	\end{figure}

	\begin{figure}
	\includegraphics[scale=0.5]{section/8.png}
	\centering
	\caption{Tahapan Instalasi 8}
	\end{figure}

	\begin{figure}
	\includegraphics[scale=0.5]{section/9.png}
	\centering
	\caption{Tahapan Instalasi 9}
	\end{figure}

\subsection{Intalasi PIP}
Langkah-langkah mengisntall PIP

	\begin{figure}
	\includegraphics[scale=0.5]{section/pip1}
	\centering
	\caption{Ketik printah : conda install -c anaconda pip}
	\end{figure}
	
	\begin{figure}
	\includegraphics[scale=0.5]{section/pip2}
	\centering
	\caption{Tunggu hingga selesai}
	\end{figure}
	
\section{Setting Environment}
Langkah-langkah seperti ini : 

	\begin{figure}
	\includegraphics[scale=0.5]{section/envirotment1}
	\centering
	\caption{pilih advanced system setting}
	\end{figure}	
	
	\begin{figure}
	\includegraphics[scale=0.5]{section/envirotment2}
	\centering
	\caption{pilih envirotmet variabel}
	\end{figure}
	
	\begin{figure}
	\includegraphics[scale=0.5]{section/envirotment3}
	\centering
	\caption{setting environment}
	\end{figure}
\section{Entrepreter atau CLI melalui terminal atau cmd windows}
pada tahap ini, dibutuhkanya cmd sebagai bahan pembelajaran dari mulai cek status python yang sudah terbaru, hingga proses pengupdatetan seperti contoh dibawah ini:

	\begin{figure}
	\includegraphics[scale=0.5]{section/13.png}
	\centering
	\caption{Tahapan Cek python}
	\end{figure}

	\begin{figure}
	\includegraphics[scale=0.5]{section/14.png}
	\centering
	\caption{Tahapan pengudatetan}
	\end{figure}

\section{Mengupdate Spyder}
Pada langkah ini, dibutuhkan sebuah command Prompt dengan megetikkan
\\ conda install -c anaconda spyder

	\begin{figure}
	\includegraphics[scale=0.5]{section/10.png}
	\centering
	\caption{Tahapan update spyder}
	\end{figure}

\section{Menjalankan Hello World}
Pada langkah ini harus di siapkan spyder yang digunakan sebagai text editor yang membantu menerjemahkan bahasa pemograman python, diantaranya sebagai berikut :

	\begin{figure}
	\includegraphics[scale=0.5]{section/11.png}
	\centering
	\caption{Syntak hello world}
	\end{figure}

	\begin{figure}
	\includegraphics[scale=0.5]{section/12.png}
	\centering
	\caption{menampilkan hello world}
	\end{figure}


\section{Menjalankan Script Otomatis Login Aplikasi Akademik}
Pada langkah pertama instal selenium terlebih dahulu seperti contoh ini :

	\begin{figure}
	\includegraphics[scale=0.5]{section/selenium.png}
	\centering
	\caption{Install selenium}
	\end{figure}

	\begin{figure}
	\includegraphics[scale=0.5]{section/proses_selenium}
	\centering
	\caption{Tahap instalasi selenium}
	\end{figure}
	
	\begin{figure}
	\includegraphics[scale=0.5]{section/login_otomatis}
	\centering
	\caption{Syntak pada spyder untuk otomatis login}
	\end{figure}
	
	\begin{figure}
	\includegraphics[scale=0.3]{section/hasil.png}
	\centering
	\caption{hasil}
	\end{figure}

\section{Variable Explorer}
Variabel explorer digunakan sebagai bawaan untuk mengedit daftar, string, kamus, array NumPy, Pandas DataFrames, dan banyak lagi, dan dapat juga histogram, plot, atau bahkan menampilkan beberapa di antaranya sebagai gambar RGB. Bisa dicek dengan mengklik variabel explorer pada spyder.

	\begin{figure}
	\includegraphics[scale=0.3]{section/variabel_explorer.png}
	\centering
	\caption{cek variabel explorer}
	\end{figure}
	
\section{Identasi}
Indentasi adalah bagian paragraf yang menjorok ke dalam pada baris-baris
paragraf, penulisan kode python tidak memakai curly brackets ”{}” sehingga
cara membedakan blok program digunakan identasi.
jenis error identasi yaitu IndentationError: expected an indented block.
artinya ini berarti fungsi if memerlukan indentasi untuk membedakan blok
kode.

	\begin{figure}
	\includegraphics[scale=0.5]{section/salah.png}
	\centering
	\caption{syntak eror identasi}
	\end{figure}
	
	\begin{figure}
	\includegraphics[scale=1.5]{section/benar.png}
	\centering
	\caption{cek variabel explorer}
	\end{figure}
\chapter*{Perusahaan Dunia yang menggunakan bahasa pemrograman Python}

\section*{\textit{Spotify}}
\par
\textit{Spotify} merupakan layanan musik streaming yang sudah banyak digunakan di seluruh dunia. Dalam bidang menganalisis data spotify menggunakan Bahasa Pemrograman Python, dalam pengimplementasiannya Tim Spotify menggunakan Luigi, modul yang ada di Python yang disingkronisasikan pada sebuah software yang memudahkan programmer membuat aplikas web atau disebut framework yang berbasis Java yang memungkinkan pemrosesan data dalam waktu cepat.Penerapan bahasa Python juga digunakan dalam penerapan fitur Radio dan Discover serta fitur merekomendasikan orang yang mungkin akan diikuti.

\section*{\textit{Netflix}}
\par
\textit{Netflix} merupakan layanan pemutaran film atau tayangan yang memungkinkan para penggunaknya menggunakan di manapun dan kapanpun. Netflix menggunakan bahasa pemrograman Python. Penggunaan Python di Netflix terdapat pada Central Alert Gateaway (C.A.G) ini akan me-reroute alert dan mengirimkannya pada kelompok atau individu yang dapat melihatnya dan secara otomatis reboot atau menghentikan proses yang dianggap bermasalah dan digunakan untuk menulusuri riwayat dan perubahan pengaturan keamanan. Tetapi sama halnya seperti spotify, Netflix menggunakan python untuk menganilisis data dan lebih utama terlihat pada bagian bagaimana netflix merekomendasikan film kepada pelangganya.

\section*{\textit{Pinterest}}
\par
\textit{Pinterest} adalah aplikasi web yang digunakan untuk mengumpulkan hal-hal yang menarik berdasarkan kriteria tertentu yang sering dikunjungi di jaman sekarang. Pinterest menggunakan Bahasa Pemrograman Python dari awal mereka membangunnya itulah sebabnya bookmarking (Sebuah metode bagi pengguna internet untuk mengorganisasi, menyimpan, mengelola, dan mecari penanda sumber daya yang tersedia secara online) yang ada di pinterest begitu terstruktur dan mudah untuk diatur.

\section*{\textit{Instagram}}
\par
\textit{Instagram} merupakan sebuah aplikasi berbagi foto dan video secara digital yang digunakan oleh lebih dari 400 juta user yang aktif setiap harinya. Instagram menggunakan bahasa pemrograman python dalam task queuenya atau fitur dimana pada saat yang bersamaan instagram dapat melakukan posting ke beberapa social network lainnya seperti Facebook, Twitter, dll. 

\section*{\textit{Industrial Light and Magic}}
\textit{Indusrial Light and Magic} adalah Studio spesial-efek yang digunakan pada pemutaran efek di film Star Wars. Dalam pembuatan efek ledakan ILM menggunakan Python dikarenakan dapat menghemat waktu dalam pembuatan efek tersebut.   
\par


\chapter*{instalasi}


\section*{\textit{ Instalasi Python 3}}
\begin{enumerate}
		\item Clik apk anaconda lau clik install. Selanjutnya clik next.
		\begin{figure}[h]
			\includegraphics[width=4cm]{figure/1.png}
			\centering
			\caption{install anaconda}
			\end{figure}
		\item Selanjutnya , clik I agree.
			\begin{figure}[h]
			\includegraphics[width=4cm]{figure/2.png}
			\centering
			\caption{licence agreement}
			\end{figure}
		\item Pilih Just me.
			\begin{figure}[h]
			\includegraphics[width=4cm]{figure/3.png}
			\centering
			\caption{installation type}
			\end{figure}
		\item Pilih lokasi penyimpanan yang akan diinstal, lalu clik next.
			\begin{figure}[h]
			\includegraphics[width=4cm]{figure/4.png}
			\centering
			\caption{lokasi penyimpanan file anaconda}
			\end{figure}
		\item Ceklis bagian ADD Environtment to the Path, hal ini memungkinkan untuk menambahkan environtment anaconda ke dalam path yang ada dalam PC anda. Setelah itu klik next.
			\begin{figure}[h]
			\includegraphics[width=4cm]{figure/5.png}
			\centering
			\caption{menambahkan path environtment}
			\end{figure}
		\item Tunggu instalan sampai selesai.
			\begin{figure}[h]
			\includegraphics[width=4cm]{figure/7.png}
			\centering
			\caption{proses instalasi}
			\end{figure}
		\item Setelah Instal selesai, lau clik next sampai proses terakhir dan clik finish di akhir proses instal.
			\begin{figure}[h]
			\includegraphics[width=4cm]{figure/8.png}
			\centering
			\caption{instalasi selesai}
			\end{figure}
		\item Click next.	
			\begin{figure}[h]
			\includegraphics[width=4cm]{figure/9.png}
			\centering
			\caption{instalasi selesai 2}
			\end{figure}
		\item Click next lagi.	
			\begin{figure}[h]
			\includegraphics[width=4cm]{figure/10.png}
			\centering
			\caption{instalasi selesai 3}
			\end{figure}
		\item Lalu click finish.	
	\end{enumerate}

\section*{\textit{ Instalasi Pip }}
\begin{enumerate}
\item Buka Cmd (CommandPrompt)
\begin{figure}[h]
\includegraphics[width=4cm]{figure/pip1.png}
\centering
\caption{Buka Cmd}
\end{figure}
\item masuk kedalam direktori yang ada file pip nya.
\begin{figure}[h]
\includegraphics[width=4cm]{figure/pip2.png}
\centering
\caption{Masuk kedalam Direktori}
\end{figure}
\item Ketikan syntax untuk install pip nya
\begin{figure}[h]
\includegraphics[width=4cm]{figure/pip3.png}
\centering
\caption{Mengetik Syntax}
\end{figure}
\item Jika sudah ada bacaan successfuly instaled, proses instalasi sudah selesai
\begin{figure}[h]
\includegraphics[width=4cm]{figure/pip4.png}
\centering
\caption{Selesai Install}
\end{figure}
\end{enumerate}

\section*{\textit{ Cara setting environment }}

\begin{enumerate}

\item Buka file exploler, lal click kanan di this pc, pilih properties
\begin{figure}[h]
\centering
\includegraphics[scale=0.8]{figure/setting1.png}
\caption{membuka file exploler}
\end{figure}

\item Buka advance system settings
\begin{figure}[hb]
\centering
\includegraphics[scale=0.8]{figure/setting2.png}
\caption{membuka advance system}
\end{figure}

\item Click environment variables
\begin{figure}[hb]
\includegraphics[width=4cm]{figure/setting3.png}
\centering
\caption{membuka environtment variables}
\end{figure}

\item Jika ingin mensetting pilih editt
\begin{figure}[h]
\includegraphics[width=4cm]{figure/setting4.png}
\centering
\caption{mensetting}
\end{figure}

\section*{\textit{ mencoba entrepreter/cli melakui terminal atau cmd windows }}

\begin{enumerate}

\item Buka Cmd
\begin{figure}[hb]
\includegraphics[width=4cm]{figure/interpreter1.png}
\centering
\caption{membuka Cmd}
\end{figure}

\item Menuliskan syntax
\begin{figure}[hb]
\includegraphics[width=4cm]{figure/interpreter2.png}
\centering
\caption{menuliskan syntax python}
\end{figure}

\item Menuliskan syntax
\begin{figure}[hb]
\includegraphics[width=4cm]{figure/setting3.png}
\centering
\caption{menuliskan syntax}
\end{figure}

\item melihat hasil syntax
\begin{figure}[hb]
\includegraphics[width=4cm]{figure/setting4.png}
\centering
\caption{melihat hasil}
\end{figure}

\end{enumerate}

\section*{\textit{ Menjalankan dan mengupdate anaconda dan spyder }}

\begin{enumerate}

\item Buka aplikasi anaconda, lalu pilih environment.
\begin{figure}[hb]
\includegraphics[width=4cm]{figure/update1.png}
\centering
\caption{membuka aplikasi}
\end{figure}

\item Menuliskan spyder di search package
\begin{figure}[hb]
\includegraphics[width=4cm]{figure/update2.png}
\centering
\caption{search package}
\end{figure}

\item Click kanan pada tanda centang pada bacaan spyder, lalu pilih mark for specific version instalation.
\begin{figure}[hb]
\includegraphics[width=4cm]{figure/update3.png}
\centering
\caption{memilih versi yang akan di install}
\end{figure}

\item Sama seperti yang tadi, search package yang akan di install lalu pilih versi yang akan di up.
\begin{figure}[hb]
\includegraphics[width=4cm]{figure/update5.png}
\centering
\caption{mengupdate anaconda}
\end{figure}

\end{enumerate}

\section*{\textit{ Cara menjalankan Script hello word di spyder }}

\begin{enumerate}

\item Buka aplikasi spyder.
\begin{figure}[hb]
\includegraphics[width=4cm]{figure/pemakaian1.png}
\centering
\caption{membuka aplikasi}
\end{figure}

\item Menuliskan syntax 
\begin{figure}[hb]
\includegraphics[width=4cm]{figure/pemakaian2.png}
\centering
\caption{Menuliskan syntax}
\end{figure}

\item Hasil syntax.
\begin{figure}[hb]
\includegraphics[width=4cm]{figure/pemakaian3.png}
\centering
\caption{Hasil syntax}
\end{figure}

\end{enumerate}

\section*{\textit{Cara menjalankan Script otomatis login aplikasi akademik dengan library selenium dan inputan user}}

\begin{enumerate}

\item Buka aplikasi spyder.
\begin{figure}[hb]
\includegraphics[width=4cm]{figure/pemakaian1.png}
\centering
\caption{membuka aplikasi}
\end{figure}

\item Menuliskan syntax 
\begin{figure}[hb]
\includegraphics[width=4cm]{figure/login.png}
\centering
\caption{Menuliskan syntax}
\end{figure}

\item Hasil syntax.
\begin{figure}[hb]
\includegraphics[width=4cm]{figure/login1.png}
\centering
\caption{Hasil syntax}
\end{figure}

\end{enumerate}


\section*{\textit{ Cara menjalankan Script otomatis login aplikasi akademik dengan library sele-
nium dan inputan user }}

\begin{enumerate}

\item Buka aplikasi spyder.
\begin{figure}[hb]
\includegraphics[width=4cm]{figure/pemakaian1.png}
\centering
\caption{membuka aplikasi}
\end{figure}

\item Menuliskan syntax 
\begin{figure}[hb]
\includegraphics[width=4cm]{figure/login.png}
\centering
\caption{Menuliskan syntax}
\end{figure}

\item Hasil syntax.
\begin{figure}[hb]
\includegraphics[width=4cm]{figure/login1.png}
\centering
\caption{Hasil syntax}
\end{figure}

\end{enumerate}

\section*{\textit{ Cara pemakaian variable explorer di spyder }}

\begin{enumerate}

\item Buka aplikasi spyder.
\begin{figure}[hb]
\includegraphics[width=4cm]{figure/pemakaian1.png}
\centering
\caption{membuka aplikasi}
\end{figure}

\item Menuliskan syntax 
\begin{figure}[hb]
\includegraphics[width=4cm]{figure/pemakaian4.png}
\centering
\caption{Menuliskan syntax}
\end{figure}

\item Hasil syntax.menunjukan nama variable, type, dan value
\begin{figure}[hb]
\includegraphics[width=4cm]{figure/pemakaian5.png}
\centering
\caption{Hasil syntax}
\end{figure}

\end{enumerate}






%now enable appendix numbering format and include any appendices
\appendix


%next line adds the Bibliography to the contents page

\end{document}

