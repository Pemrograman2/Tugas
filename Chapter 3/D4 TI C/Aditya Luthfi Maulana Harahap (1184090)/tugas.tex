\documentclass[12pt, times new roman]{report}
\usepackage[utf8]{inputenc}
\usepackage{color}
\usepackage{listings}

\definecolor{codegreen}{rgb}{0,0.6,0}
\definecolor{codegray}{rgb}{0.5,0.5,0.5}
\definecolor{codepurple}{rgb}{0.58,0,0.82}
\definecolor{backcolour}{rgb}{0.95,0.95,0.92}

\lstdefinestyle{mystyle}{
    backgroundcolor=\color{backcolour},   
    commentstyle=\color{codegreen},
    keywordstyle=\color{magenta},
    numberstyle=\tiny\color{codegray},
    stringstyle=\color{codepurple},
    basicstyle=\footnotesize,
    breakatwhitespace=false,         
    breaklines=true,                 
    captionpos=b,                    
    keepspaces=true,                 
    numbers=left,                    
    numbersep=5pt,                  
    showspaces=false,                
    showstringspaces=false,
    showtabs=false,                  
    tabsize=2,
    language=python
}

\lstset{style=mystyle}

\title{Tugas Pemrograman Chapter 3}
\author{Aditya Luthfi Maulana Harahap (1184090)}
\date{\today}

\begin{document}

\maketitle

\chapter{Teori}

\section{Fungsi}
\hspace{1cm} Fungsi adalah bagian dari program yang dapat digunakan ulang. Hal ini bisa dicapai dengan memberi nama pada blok statemen, kemudian nama ini dapat dipanggil di manapun dalam program.

\lstinputlisting[language=python]{src/fungsi.py}

\section{Library}
\hspace{1cm} Paket(library) adalah sebuah file yang berisi kode pemrograman python. Library tidak lain adalah program python biasa. Berikut ini kita mencoba membuat sebuah library. Kita akan menyimpannya sebagai library.py :

\lstinputlisting[language=python]{src/library.py}

\hspace{0.5cm} Contoh pemanggilan library yang sudah kita buat di program baru fungsi1.py harus menggunakan import :

\lstinputlisting[language=python]{src/fungsi1.py}


\section{Kelas, Objek, Atribut, Method}

\subsection{Kelas}
\hspace{1cm} Kelas adalah sebuah blueprint untuk mendefinisikan suatu objek.
\lstinputlisting[language=python]{src/class.py}

\subsection{Objek}
\hspace{1cm} Objek adalah hasil blueprint dari sebuah kelas.
\lstinputlisting[language=python]{src/objek.py}

\subsection{Atribut}
\hspace{1cm} Atribut adalah karakteristik yang dimiliki oleh sebuah objek.
\lstinputlisting[language=python]{src/atribut.py}

\subsection{Method}
\hspace{1cm} Method merupakan sebuah kumpulan code program yang digunakan untuk melakukan suatu perintah. Dan bedanya dengan baris code program biasa adalah, kita dapat melakukan hal yang sama cukup dengan  memanggil nama metode nya saja.. Tanpa harus menuliskan code program yang ingin kita ulang
\lstinputlisting[language=python]{src/method.py}

\section{ pemanggilan library dalam sebuah folder}
\hspace{1cm} Pertama-tama ciptakan library yang akan digunakan contoh :

\lstinputlisting[language=python]{src/library.py}

\hspace{0.5cm} Lalu cara untuk memanggilnya seperti ini :

\lstinputlisting[language=python]{src/fungsi1.py}

\section{library dengan perintah from kalkulator import tambah}
\hspace{1cm} Terlebih dahulu ciptakan library dengan nama tambah yang di letakkan di folder kalkulator

\lstinputlisting[language=python]{src/kalkulator/tambah.py}

\hspace{0.5cm} Lalu ciptakan program hitung 

\lstinputlisting[language=python]{src/hitung.py}

\section{Pemakaian paket fungsi jika file library berada di dalam folder}
\hspace{1cm} Jika kondisinya seperti ini maka gunaka perintah from 'nama folder' import 'nama library' di program yang ingin menggunakan library tersebut.

\lstinputlisting[language=python]{src/hitung.py}

\hspace{0.5cm} Ini artinya kita memanggil fungsi tambah yang berada di folder kalkulator

\section{Pemakaian paket kelas jika file library berada di dalam folder}
\hspace{1cm} Jika kondisinya seperti ini maka gunaka perintah from 'nama folder' import 'nama library' di program yang ingin menggunakan library tersebut.

\lstinputlisting[language=python]{src/hitung.py}

\hspace{0.5cm} Ini artinya kita memanggil kelas tambah yang berada di folder kalkulator

\chapter{Keterampilan Pemrograman}

\section*{Soal 1}
\lstinputlisting[language=python]{src/soal1.py}

\section*{Soal 2}
\lstinputlisting[language=python]{src/soal2.py}

\section*{Soal 3}
\lstinputlisting[language=python]{src/soal3.py}

\section*{Soal 4}
\lstinputlisting[language=python]{src/soal4.py}

\section*{Soal 5}
\lstinputlisting[language=python]{src/soal5.py}

\section*{Soal 6}
\lstinputlisting[language=python]{src/soal6.py}

\section*{Soal 7}
\lstinputlisting[language=python]{src/soal7.py}

\section*{Soal 8}
\lstinputlisting[language=python]{src/soal8.py}\

\section*{Soal 9}
\lstinputlisting[language=python]{src/soal9.py}

\section*{Soal 10}
\lstinputlisting[language=python]{src/soal10.py}

\section*{Soal 11}
\lstinputlisting[language=python]{src/lib3.py}

\section*{Soal 12}
\subsection*{kelas3lib.py}
\lstinputlisting[language=python]{src/kelas3lib.py}
\subsection*{main.py}
\lstinputlisting[language=python]{src/main.py}

\chapter{Keterampilan Penanganan error}

\section*{Penanganan Erorr}

error:\\
TypeError: \textunderscore\textunderscore init\textunderscore\textunderscore () missing 1 required\\ positional argumen: 'npm'\\
menanambahkan parameter pada fungsi\\


\section*{Try except}
\lstinputlisting[language=python]{src/try.py}
\end{document}