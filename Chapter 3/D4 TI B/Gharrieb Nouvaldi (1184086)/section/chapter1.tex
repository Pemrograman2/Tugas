\chapter*{Teori}

\begin{enumerate}
	\item Fungsi merupakan blok program pada Python, dibuat dengan menggunakan kata kunci def kemudian diikuti dengan nama fungsinya. Sama seperti blok kode yang lain, kita harus memberikan identasi (tab atau spasi 2x) untuk menuliskan isi fungsi.\\
	Contoh pengaplikasian dari fungsi:\\
	\#pembuatan fungsi\\
	  def Salam():  
        print "Hay, Selamat Pagi"\\
	 \#pemanggilan fungsi \\ 
	    salam()\\
	  \#output: Selamat Pagi \\
	
	\item Package adalah folder yang menyimpan source code dan dapat di import kedalam program\\
	\#Cara memanggilnya : from nama\_package import library.
	
	\item a). Class adalah blue print  untuk mendefinisikan suatu objek. Contoh kode program: \\
	Class NPM :\\
	def\_\_ init\_\_(self, NPM):\\
		self.NPM = NPM\\
	def  helo NPM (self):\\
		print (“Helo”, NPM)\\
	b). Objek adalah hasil dari sebuah class.Contohnya :\\
	Import kelas2B
	
	Cobakelas=kelas2B.aku(npm)\\
 	Hasil=Cobakelas.npm2()\\
	c).	Atribut merupakan  Fungsi-fungsi yang dimiliki oleh class. Contoh penggunaanya:\\
	Class name:\\
	def \_\_ init \_\_(self,nama):\\
	Self.nama = nama\\
	d). Method adalah Fungsi yang ada pada objek atau intansi class. Contoh penggunaanya:\\
	Class name:\\
	def \_\_ init \_\_(self,nama):
	Self.nama = nama\\
	def nama(self)\\
	print(“hallo”, name)\\
	
	\item Contoh penggunakan sebuah library : pada folder Gharrieb\\
	def  nama () :\\
	print (“Gharrieb”)\\
	Contoh jika ingin memanggil fungsi dari library, maka kita harus import terlebih dahulu:\\
	Import Gharrieb\\
	Gharrieb.nama ()\\
	
	\item Perintah form kalkulator penambahan.\\
	Maksudnya dari from kalkulator import penambahan yaitu, kode tersebut akan memanggil kalkulator dan mengimport penambahan.\\
	Contohnya : from Mahasiswa import NPM\\
	
	\item Cara memanggil library dalam folder.\\
	Langkah pertama kita harus menuliskan foldernya terlebih dahulu lalu melakukan import nama librarynya.\\
    Contohnya: from Mahasiswa import NPM\\ 
	
	\item Cara memanggil class di dalam folder\\
	Cara melakukan pemanggilannya dengan cara memanggil foldernya terlebih dahulu baru setelah itu library yang kita inginkan. Contohnya:\\
	from Mahasiswa import NPM\\
	nama kelas yang akan digunakan adalah NPM\\
	


\end{enumerate}
