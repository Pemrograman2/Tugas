\documentclass[a4paper,12pt]{report}
\usepackage{graphicx}
\usepackage{listings}

\usepackage{color}
 
\definecolor{codegreen}{rgb}{0,0.6,0}
\definecolor{codegray}{rgb}{0.5,0.5,0.5}
\definecolor{codepurple}{rgb}{0.58,0,0.82}
\definecolor{backcolour}{rgb}{0.95,0.95,0.92}
 
\lstdefinestyle{mystyle}{
    backgroundcolor=\color{backcolour},   
    commentstyle=\color{codegreen},
    keywordstyle=\color{magenta},
    numberstyle=\tiny\color{codegray},
    stringstyle=\color{codepurple},
    basicstyle=\footnotesize,
    breakatwhitespace=false,         
    breaklines=true,                 
    captionpos=b,                    
    keepspaces=true,                 
    numbers=left,                    
    numbersep=5pt,                  
    showspaces=false,                
    showstringspaces=false,
    showtabs=false,                  
    tabsize=2,
    language=python
}
 
\lstset{style=mystyle}

\title{Tugas pemrograman 2}
\author{idam fadilah}
\date{23 oktober 2019}
\begin{document}
\maketitle
\chapter{Python}
\section{Fungsi}
\paragraph{}
Fungsi adalah blok kode yang akan dieksekusi ketika dipanggil dalam sebuah program
\subsection*{Parameter}
parameter yaitu inputan sebuah fungsiyang bertujuan untuk menyimpan sebuah nilai
\subsection*{Return}
return digunakan untuk mengembalikan sebuah nilai, atau bisa juga mengakhiri eksekusi sebuah fungsi

\begin{lstlisting)
\end
\section{Package}
\paragraph{}
package merupakan sebuah folder yang menyimpan source code
\section{Class}
class merupakan blueprint dari sebuah object, jika diibaratkan membuat sebuah kue, class merupakan cetakan kuenya, contoh kode :
\begin{lstlisting}[language=Python]
class Kelas3ngitung:
    def __init__(self,npm):
        self.npm = npm
    def npm1(self):
        return lib3.npm1()
    def npm2(self):
        return lib3.npm2(self.npm)
\end{lstlisting}
\subsection*{Object}
object merupakan hasil cetakan dari sebuah class
contoh kode :
\begin{lstlisting}[language=Python]
#import kelas terlebih dahulu
import kelas3lib
#membuat object
cobakelas=kelas3lib.Kelas3ngitung(npm) 
hasilkelas=cobakelas.npm1()
\end{lstlisting}
\subsection*{Atrribute}
attribute merupakan variabel global yang dimiliki oleh sebuah class
\begin{lstlisting}[language=Python]
class Kelas3ngitung:
	#pendefinisian attribute
    def __init__(self,npm):
        self.npm = npm
    def npm1(self):
        return lib3.npm1()
    def npm2(self):
        return lib3.npm2(self.npm)
\end{lstlisting}
\subsection*{Method}
\begin{lstlisting}[language=Python]
class Kelas3ngitung:
    def __init__(self,npm):
        self.npm = npm
    #Pembuatan method pada class
    def npm1(self):
        return lib3.npm1()
    def npm2(self):
        return lib3.npm2(self.npm)
\end{lstlisting}

\begin{lstlisting}[language=Python]
import kelas3lib
cobakelas=kelas3lib.Kelas3ngitung(npm) 
#pemanggilan method pada program
hasilkelas=cobakelas.npm1()
\end{lstlisting}

\chapter{Keterampilan pemrograman}
\section*{Soal 1}
\lstinputlisting[language=Python]{src/npm1.py}
\section*{Soal 2}
\lstinputlisting[language=Python]{src/npm2.py}
\section*{Soal 3}
\lstinputlisting[language=Python]{src/npm3.py}
\section*{Soal 4}
\lstinputlisting[language=Python]{src/npm4.py}
\section*{Soal 5}
\lstinputlisting[language=Python]{src/npm5.py}
\section*{Soal 6}
\lstinputlisting[language=Python]{src/npm6.py}
\section*{Soal 7}
\lstinputlisting[language=Python]{src/npm7.py}
\section*{Soal 8}
\lstinputlisting[language=Python]{src/npm8.py}
\section*{Soal 9}
\lstinputlisting[language=Python]{src/npm9.py}
\section*{Soal 10}
\lstinputlisting[language=Python]{src/npm10.py}
\section*{Soal 11}
\lstinputlisting[language=Python]{src/lib3.py}
\section*{Soal 12}
\lstinputlisting[language=Python]{src/kelas3lib.py}

\section*{main.py}
\lstinputlisting[language=Python]{src/main.py}

\chapter{Keterampilan penanganan error}
\section*{}







\end{document}