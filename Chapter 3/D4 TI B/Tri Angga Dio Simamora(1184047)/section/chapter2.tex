\chapter{Keterampilan Pemrograman}

\section{Listing Program Code}

\subsection{Code 1}
Fungsi init sebagai constructor dari variabel npm
\lstinputlisting[language=Python, firstline=3, lastline=4]{src/kode_tugas.py}

\lstinputlisting[language=Python, firstline=6, lastline=19]{src/kode_tugas.py}

\lstinputlisting[language=Python, firstline=21, lastline=29]{src/kode_tugas.py}

\lstinputlisting[language=Python, firstline=31, lastline=43]{src/kode_tugas.py}

\lstinputlisting[language=Python, firstline=45, lastline=52]{src/kode_tugas.py}

\lstinputlisting[language=Python, firstline=54, lastline=64]{src/kode_tugas.py}

\lstinputlisting[language=Python, firstline=66, lastline=77]{src/kode_tugas.py}

\lstinputlisting[language=Python, firstline=79, lastline=90]{src/kode_tugas.py}

\lstinputlisting[language=Python, firstline=92, lastline=97]{src/kode_tugas.py}

\lstinputlisting[language=Python, firstline=99, lastline=108]{src/kode_tugas.py}

\lstinputlisting[language=Python, firstline=110, lastline=119]{src/kode_tugas.py}

\lstinputlisting[language=Python, firstline=121, lastline=134]{src/kode_tugas.py}

\subsection{Code 2}
