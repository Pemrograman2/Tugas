\chapter{Tugas}
\section*{Soal 1}
\begin{lstlisting}[language=Python]
def Npm(npm):
    mod = int(npm) % 3 
    if mod == 0:
        print("*")
    elif mod == 1:
        print("#")
    elif mod == 2:
        print("+")


print("******   ******    *********   ***   ***  *********  ********* ********* ")
print("******   ******    ***   ***   ***   ***  ***   ***  ***       ***   *** ")
print("  ****     ****    *********   *********  ***   ***  ********* ********* ")
print("  ****     ****    *********   *********  ***   ***  ***   ***       *** ")
print("******** ********  ***   ***        ****  ***   ***  ***   ***       *** ")
print("******** ********  *********        ****  *********  ********* ********* ")

\end{lstlisting}

\section*{Soal 2}
\begin{lstlisting}[language=Python]
def ulang(npm):
    mod = int(npm) % 100
    for i in range(mod):
        print("hallo",npm,"apa kabar ?")

\end{lstlisting}

\section*{Soal 3}
\begin{lstlisting}[language=Python]
def ulangplus(npm):
    mod = int(npm)%1000
    string = str(mod)
    sub = npm[4] + npm[5] + npm[6]
    
    for i in range(int(npm[4])+int(npm[5])+int(npm[6])):
        print("hallo",npm[4] + npm[5] + npm[6],"apa kabar ?")


\end{lstlisting}
\section*{Soal 4}
\begin{lstlisting}[language=Python]

def ulangnol(npm):
    sub = npm[4]
        
    print("hallo",npm[4],"apa kabar")

\end{lstlisting}
\section*{Soal 5}
\begin{lstlisting}[language=Python]

def turun(npm):
    
    i = 0
    while i<1:
        if len(npm) < 7:
            print("npm kurang dari 7 digit, silahkan masukkan npm anda kembali")
        elif len(npm) > 7:
            print("npm yang diinputkan lebih dari 7, silahkan masukkan npm anda kembali")
        else:
            i=1
    a=npm[0]
    b=npm[1]
    c=npm[2]
    d=npm[3]
    e=npm[4]
    f=npm[5]
    g=npm[6]

    for x in a,b,c,d,e,f,g:
        print(x,""),



\end{lstlisting}
\section*{Soal 6}
\begin{lstlisting}[language=Python]

def jumlah(npm):
    
    i = 0
    while i<1:
        if len(npm) < 7:
            print("npm kurang dari 7 digit, silahkan masukkan npm anda kembali")
        elif len(npm) > 7:
            print("npm yang diinputkan lebih dari 7, silahkan masukkan npm anda kembali")
        else:
            i=1
    a=npm[0]
    b=npm[1]
    c=npm[2]
    d=npm[3]
    e=npm[4]
    f=npm[5]
    g=npm[6]
    y=0

    for x in a,b,c,d,e,f,g:
        y+=int(x)
    print(y)




\end{lstlisting}
\section*{Soal 7}
\begin{lstlisting}[language=Python]


def kali(npm):
    
    i = 0
    while i<1:
        if len(npm) < 7:
            print("npm kurang dari 7 digit, silahkan masukkan npm anda kembali")
        elif len(npm) > 7:
            print("npm yang diinputkan lebih dari 7, silahkan masukkan npm anda kembali")
        else:
            i=1
    a=npm[0]
    b=npm[1]
    c=npm[2]
    d=npm[3]
    e=npm[4]
    f=npm[5]
    g=npm[6]
    y=0

    for x in a,b,c,d,e,f,g:
        y*=int(x)
    print(y)

\end{lstlisting}
\section*{Soal 8}
\begin{lstlisting}[language=Python]

def genap(npm):
    
    i = 0
    while i<1:
        if len(npm) < 7:
            print("npm kurang dari 7 digit, silahkan masukkan npm anda kembali")
        elif len(npm) > 7:
            print("npm yang diinputkan lebih dari 7, silahkan masukkan npm anda kembali")
        else:
            i=1
    a=npm[0]
    b=npm[1]
    c=npm[2]
    d=npm[3]
    e=npm[4]
    f=npm[5]
    g=npm[6]

    for x in a,b,c,d,e,f,g:

        if int(x)%2==0:
            if int(x)==0:
                x=""
            print(x,end ="")


\end{lstlisting}
\section*{Soal 9}
\begin{lstlisting}[language=Python]

def ganjil(npm):
    
    i = 0
    while i<1:
        if len(npm) < 7:
            print("npm kurang dari 7 digit, silahkan masukkan npm anda kembali")
        elif len(npm) > 7:
            print("npm yang diinputkan lebih dari 7, silahkan masukkan npm anda kembali")
        else:
            i=1
    a=npm[0]
    b=npm[1]
    c=npm[2]
    d=npm[3]
    e=npm[4]
    f=npm[5]
    g=npm[6]

    for x in a,b,c,d,e,f,g:

        if int(x)%3==0:
            if int(x)==0:
                x=""
            print(x,end ="")


\end{lstlisting}
\section*{Soal 10}
\begin{lstlisting}[language=Python]

import lib3

class Kelas3lib:
    def __init__(self,npm):
        self.npm = npm
    def npm1(self):
        return lib3.Npm(self.npm)
    def npm2(self):
        return lib3.ulang(self.npm)
    def npm3(self):
        return lib3.ulangplus(self.npm)
    def npm4(self):
        return lib3.ulangnol(self.npm)
    def npm5(self):
        return lib3.turun(self.npm)
    def npm6(self):
        return lib3.jumlah(self.npm)
    def npm7(self):
        return lib3.kali(self.npm)
    def npm8(self):
        return lib3.genap(self.npm)
    def npm9(self):
        return lib3.ganjil(self.npm)
    











\end{lstlisting}
\section*{Soal 11}
\begin{lstlisting}[language=Python]

import kls3lib
import lib3


npm=input("Input NPM : ")
i=0
while i<1:
    if len(npm) < 7:
        print("NPM Kurang dari 7 digit")
        npm=input("Input NPM : ")
    elif len(npm) > 7:
        print("NPM lebih dari 7 digit")
        npm=input("Input NPM : ")
    else:
        i=1

#Contoh pemanggilan fungsi pada class
cobakelas=kls3lib.Kelas3lib(npm) 
hasilkelas=cobakelas.npm1()

print("")

#Contoh pemanggilan fungsi pada library
lib3.kali(npm)
lib3.turun(npm)


\end{lstlisting}
