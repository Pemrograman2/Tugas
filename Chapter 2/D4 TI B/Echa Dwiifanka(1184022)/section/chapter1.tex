\chapter{TEORI PEMROGRAMAN DASAR}

\begin{enumerate}
	\item  variabel merupakan tempat untuk menyimpan suatu data.variabel brsifat multible, artinya nilainya  dapat berubah-ubah. variabel memiliki beberapa jenis antara lain sebagai berikut:
	\begin{enumerate}
	        \item  variabel local, adalah variabel yang dapat diaksespada fungsi dan tempat yang variabelnya berbeda.
	        \item  variabel build-in, adalah variabel yang sudah ada pada script python.
	        \item variabel global, adalah variabel yang bisa diakses dengansemua fungsi.
	\end{enumerate}
	       
	  \item Dengan menuliskan sricpt seperti ini :                         A=input(”echa”)
            maka untuk  menampilkan  ketikan  perintah  dibawah  ini :
            print(”halo”, A ”Apa Kabar”)
    \item Operator  dasar matematika adalah  :
    \begin{enumerate}
        \item  + (pertambahan)
        \item  - (pengurangan)
        \item   / (pembagian)
        \item  x (perkalian)
    \end{enumerate}
    \item perulangan dapat dibagi beberapa jenis yaitu antara lain:
    \begin{enumerate}
        \item While loop adalah perulangan  yang selalu dieksekusi selama kondisi bernilai benar(true).
        \item For loop adalah perulangan  yang memiliki kemampuan  untuk  mengulangi item dari urutan apapun  seperti list atau  string
    \end{enumerate}
    \item kondisi juga yang terdapat pada python dapat dibagi menjadi beberapa antara lain:
    \begin{enumerate}
        \item if
        \par adalah suatu kondisi yang bernilai benar atau  salah, jika dalam statmennt bernilai  benar  akan  dijalankan,  tetapi  sebaliknya  jika statmennt bernilai salah maka tidak  akan dijalankan  (eror).
        \item If - Else
        \par adalah  suatu  kondisi bernilai benar  maka statment didalam  if akan diek- sekusi dan jika bernilai false maka statment yang dieksekusi adalah  stat- ment didalam else
        \item If - Elif - Else
        \par adalah  suatu  kondisi Elif, lanjutan dari percabangan  if dengan kondisi ini menyebabkan  beberapa  kemungkinan  statment yang terjadi.
    \end{enumerate}
    \item diantara error yang ditemui, berikut beberapa langkah cara mengatasi error antara lain sebagai berikut:
    \begin{enumerate}
        \item TypeError:can only concatenate  str(not”int”) to str
            penanganan  ini ditandai  dengan menggunakan casting operand kedua men- jadi string.
        \item TypeError:unsupported operand  type(s) for +:’int’  and ’str’
        penangann  error  ini bisa  ditandai   menggunakan  casting  operand  kedua menjadi integer.
    \end{enumerate}
    \item Try  Except  adalah  bentuk  penanganan  error yang terdapat dalam  bahasa  pemograman  (python).contohtry except :
    \par Menangani error pembacaan file
    \par orang = {"nama":"syuaib", "kota":"jepara", "umur":"20"}

    \par    try:
         \par contact = open("contact.txt", 'r')
        \par print orang["pekerjaan"]
         \par except IOError, e:
         \par print "Terjadi error IO: ", e
        \par except KeyError, e:
         \par print "Terjadi kesalahan pada akses list/dict/tuple:", e
         \par print orang
    \par maka outputnya sebagai berikut:
    \par python demo-2.py 
       \par  Terjadi error IO:  [Errno 2] No such file or directory: '/home/contact.txt'
        \par {{'nama': 'syuaib', 'umur': '20', 'kota': 'jepara'}}
        
\chapter{KETERAMPILAN PEMROGRAMAN}
\section*{soal 1}\\
\begin{lstlisting}
# -*- coding: utf-8 -*-
"""
Created on Tue Oct 29 18:23:07 2019

@author: USER
"""

print("***   ***   *******   ******  ****** ******  ******");
print("***   ***   **   **  **   **  **  **     **      **");
print("***   ***    ****   ********  **  ** ******  ******");
print("***   ***   **   **       **  **  ** **      **"    );
print("***   ***   *******       **  ****** ******  ******");
\end{lstlisting}
\section*{soal 2}\\
\begin{lstlistening}
# -*- coding: utf-8 -*-\\
"""\\
Created on Tue Oct 29 18:38:00 2019\\

@author: USER\\
"""\\

npm=int(input("masukan npm anda : "))\\
TwoLastDigit=abs(npm)%100 # modulus menetukan ambil 2 digit terakhir\\
for i in range(TwoLastDigit):\\
    print("Halo, ", npm, " Apa kabar ?")\\
\end{lstlistening}
\section*{soal 3}\\
\begin{lstlistening}
# -*- coding: utf-8 -*-\\
"""\\
Created on Tue Oct 29 18:47:48 2019\\

@author: USER\\
"""\\

npm=int\\
key=str(npm%1000)\\
print("Hallo, "+str(npm)[4]+str(npm)[5]+str(npm)[6]+" Apa kabar?")\\

for i in range(int(str(npm)[4])+int(str(npm)[5])+int(str(npm)[6])-1):\\
         print("Hallo, "+str(npm)[4]+str(npm)[5]+str(npm)[6]+" Apa kabar?")\\
\end{lstlistening}
\section*{soal 4}
\lstinputlisting[language=Python]{src/NPM(4).py}
\section*{soal 5}
\lstinputlisting[language=Python]{src/NPM(5).py}
\section*{soal 6}
\lstinputlisting[language=Python]{src/NPM(6).py}
\section*{soal 7}
\lstinputlisting[language=Python]{src/NPM(7).py}
\section*{soal 8}
\lstinputlisting[language=Python]{src/NPM(8).py}
\section*{soal 9}
\lstinputlisting[language=Python]{src/NPM(9).py}
\section*{soal 10}
\lstinputlisting[language=Python]{src/NPM(10).py}
\section*{soal 11}
\lstinputlisting[language=Python]{src/NPM(11).py}
\section*{soal 12}
\lstinputlisting[language=Python]{src/2errr.py}
 \end{enumerate}

