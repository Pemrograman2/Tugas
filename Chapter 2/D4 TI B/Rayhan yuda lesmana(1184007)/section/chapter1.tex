\chapter*{Teori}

\section*{Pengenalan variabel}
\par
Variabel adalah sebuah tempat peyimpanan sebuah nilai. Pada python mendeklarasikan pembuatan variabel akan terjadi secara otomatis, tanda = untuk memberikan nilai pada variabel.

Variabel Global pada python yitu variabel yang bisa diakses dari semua fungsi.\\
Sedangkan Variabel Lokal pada python yaitu variabel yang hanya dapat diakses didalam fungsi tempat ia berada.\\
Variabel build-in adalah variabel yang sudah terpasang pada python

Contoh kode:\\
(Membuat variabel Global)\\
nama = "rayhan"\\
versi = "1.0.0"

def help():\\
	(Ini variabel lokal)\\
	nama = "programku"\\
	versi = "1.0.2"\\
	(Mengakses variabel lokal)\\
	print("Nama :", nama)\\
	print("Versi :", versi)\\
(Memanggil fungsi help)\\
help()	\\
\section*{Inputan user dan melakukan out put ke layar}
npm=input(" NPM Kamu :") : Meminta inputan user\\
print("Hai"+npm)        : Out Put ke layar\\
Hasil:\\
\\
NPM Kamu : (Masukan npm anda)\\
Hai (npm anda)\\
\\
NPM Kamu : 1184007\\
Hai 1184007\\

\section*{Operator dasar aritmatika dan mengubah string ke integer dan juga sebaliknya }

\subsection*{Operator dasar}
\begin{enumerate}
\item Penjumlahan 	: +
\item Pengurangan 	: -
\item Perkalian 	: *
\item Pembagian 	: /
\item Sisa bagi		:\%
\item Pemangkatan	: **
\end{enumerate}
\subsection*{Casting}
\par
Casting yaitu cara untuk mengubah tipe data dari suatu data primitive.\\
Syntax untuk melakukan casting:\\
\begin{enumerate}
\item int(var/value): merubah tipe data ke integer.
\item float(var/value): merubah tipe data ke float.
\item string(var/value): merubah tipe data ke str.
\end{enumerate}
Contoh:\\
Menggunakan int()untuk mengubah tipe data variabel z menjadi int.
z="1"
y=2
int(z)+y
\section*{Perulangan}
\begin{enumerate}
\item Perulangan for\\
\par
Perulangan For yaitu perulangan yang terhitung (Counted loop), perulangan ini biasa nya dipakai
untuk mengulang kode yang sudah diketahui banyak pengulangan nya.\\

Contoh syntax:\\
ulang = 10\\
for i in range(ulang):
	print ("perulangan ke-"str(i))
\item Perulangan while\\
\par
Perulangan While yaitu perulangan yang tak terhitung (uncounted loop), perulangan ini biasa nya memiliki syarat untuk menghetikan perulangan nya dan tidak tentu banyak perulangan nya.\\

Contoh syntax:\\
jawab = 'ya'\\
hitung = 0\\
while(jawab == 'ya'):\\
	hitung +=1\\
	jawab = input("ulang lagi tidak?")\\
print("Total perulangan: ", str(hitung)\\
\end{enumerate}

\section*{Kondisi}
\begin{enumerate}
\item Kondisi IF\\
Digunakan untuk mengantisipasi program yang sedang berjalan dan menentukan tindakan yang akan di ambil sesuai dengan kondisi yang di butuhkan.\\
Contoh:\\
-Kondisi if adalah kondisi yang akan dieksekusi oleh program jika bernilai benar atau TRUE\\

nilai = 9\\

-jika kondisi benar/TRUE maka program akan mengeksekusi perintah dibawahnya\\
if(nilai > 7):\\
    print("Selamat Anda Lulus")\\
\item Kondisi IF ELSE\\
Digunakan untuk  menentukan tindakan yang akan di ambil sesuai dengan kondisi yang di butuhkan / tidak sesuai/\\
Contoh:\\
    -Kondisi if else adalah jika kondisi bernilai TRUE maka akan dieksekusi pada if, tetapi jika bernilai FALSE maka akan dieksekusi kode pada else.\\

nilai = 3\\
-Jika pernyataan pada if bernilai TRUE maka if akan dieksekusi, tetapi jika FALSE kode pada else yang akan dieksekusi.\\
if(nilai > 7):\\
    print("Selamat Anda Lulus")\\
else:\\
    print("Maaf Anda Tidak Lulus")\\
\item Kondisi Elif
Kondisi Elif ini lanjutan dari percabangan kondisi if dengan kondisi elif ini kita bisa membuat kode program yang akan menyeleksi beberapa kemungkinan yang bisa terjadi.\\
Contoh:\\
hari\_ini = "Minggu"\\

if(hari\_ini == "Senin"):\\
    print("Saya akan kuliah")\\
elif(hari\_ini == "Selasa"):\\
    print("Saya akan kuliah")\\
elif(hari\_ini == "Rabu"):\\
    print("Saya akan kuliah")\\
elif(hari\_ini == "Kamis"):\\
    print("Saya akan kuliah")\\
elif(hari\_ini == "Jumat"):\\
    print("Saya akan kuliah")\\
elif(hari\_ini == "Sabtu"):\\
    print("Saya akan kuliah")\\
elif(hari\_ini == "Minggu"):\\
    print("Saya akan libur")\\
\end{enumerate}

\section*{Jenis error}
\section*{Try Except}
\par
Try except adalah cara untuk menangani suatu eror didalam python. cara menggunakannya ialah:\\
setiap angka yang dibagi dengan 0 maka akan terjadi eror sudah ketentuannya seperti contoh dibawah ini:\\
x=0\\
try\\
x = 5/0\\

except exception, e:\\
print e\\
\\
\\
print x+1\\
maka akan muncul integer division or modulo by zero 1\\
\\
\\
seharusnya kode diatas tidak dapat dieksekusi tetapi karena menggunakan try except kode diatas dapat dieksekusi walaupun hasilnya akan eror