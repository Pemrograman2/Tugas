\documentclass{article}
\usepackage[utf8]{inputenc}

\title{Tugas Resume Pemograman II}
\author{Oleh Gany B. Sura }
\date{17 October 2019}

\begin{document}

\maketitle

\section{Apa Itu pyhton?}
\usepackage{Python merupakan bahasa pemrograman tingkat tinggi serta multiguna dimana perancangan berfokus padabagaimana kode terbaca.Bahasa pemograman Python disebut sebagai bahasa yang kemampuan yang dapat menggabungkan kapabilitas dengan sintaksis kode yang jelas dengan dilengkapi fungsionalitas pustaka standar.}\\

\usepackage{python juga merupakan bahasa pemograman scripting yang memiliki tingkat tinggi, interpreted, interactive, dan berorientais objek.  Python dibentuk dengan desain yang sangat mudah di baca dan dipahami, karena sama seperti bahasa pemograman yang lainnya yaitu dengan menggunakan kata bahasa inggris. Selain itu juga lebih sedikit dalam penggunaan rumus atau syntac.}

\section{Sejarah Python}
\usepackage{Sejarah penemuan Python berawal dari orang keturunan Belanda yaitu Guido van Rossum.Pembuatan bahasa pemograman python ini berlangsung di kota Amsterdam, Belanda pada tahun 1990. Pada tahun 1995 Python dikembangkan lagi agar lebih kompatibel oleh Guido Van Rossum. Kemudian pada awal tahun 2000, terdapat pembaharuan versi Python hingga mencapai Versi 3 sampai saat ini. nama Python sendiri diambil dari sebuah acara televisi yang lumayan terkenal yang bernama Mothy Python Flying Circus yang merupakan acara sirkus favorit dari Guido van Rossum. }
\usepackage{Awalnya, pembuatan bahasa pemrograman ini adalah untuk membuat skrip bahasa tingkat tinggi pada sebuah sistem operasi yang terdistribusi Amoeba. Python telah digunakan oleh beberapa pengembang dan bahkan digunakan oleh beberapa perusahaan untuk pembuatan perangkat lunak komersial.}

\section{Perbedaan python 2 dan python 3}
\usepackage{Pada dasarnya python dibedakan atas python 2 dan python 3, keduanya memiliki perbedaan yang tidak terlalu spesifik . Dimana Python 2 dinilai lebih transparan dan inklusif untuk pengembangan software ketimbang versi sebelumnya. Hal ini didukung dengan adanya Python Enhancement Proposal yang mana ini merupakan sebuah spesifikasi teknis yang menjadi tuntunan informasi untuk penggunanya dan menunjukan fitur baru pada Python itu sendiri. Python 2 dilengkapi juga berbagai fitur programatikal seperti cycle-detecting garbage collector untuk mengotomasikan manajemen memori, peningkatan dukungan untuk Unicode, list comprehension untuk membuat sebuah list berdasarkan list yang sudah ada. Unifikasi pada tipe data Python dan class ke satu hirarki terjadi pada rilis Python 2.2 . }
\usepackage{Sedangkan Python 3 merupakan versi dengan banyak perubahan yang hadir akhir tahun 2008. Fokus dari Python 3 itu sendiri adalah melakukan perapian pada codebase dan menghapuskan duplikasi (redundancy). Perubahan terbesar pada Python 3 termasuk memberikan statemen print ke dalam built-in function. Awalnya, Python 3 mengalami hambatan pada pengadopsiannya disebabkan tidak adanya backwards compatibility dengan Python 2. Hal ini membuat pengguna Python sangat berat hati untuk pindah ke versi 3 ini. selanjutnya banyak sekali library yang hanya tersedia untuk Python 2., tapi setelah tim pengembangan di balik Python 3 telah berulang kali menjelaskan bahwa dukungan terhadap Python 2 akan segera dihentikan, dan semakin banyak libary disalin ke Python 3, maka penerapan Python 3 semakin lama semakin meningkat dan berkembang pesat.}

\section{Impelementasi python}
\usepackage{Setelah mengetahui sejarah dan apa itu python maka kita harus mengetahui implementasi dari python di dunia kerja atau kehidupan sehari hari. Seperti yang kita ketahui Beberapa platform terkenal seperti Spotify dan Netflix adalah contoh platform yang telah memanfaatkan Python dalam analisis data. Tim Spotify memakai analitis dimana di analitis itu ada Luigi, modul dari Python, yang disinkronisasi atau dihubungkan dengan Hadoop, sebuah framework berbasis Java yang memungkinkan pemrosesan data dengan ukuran sangat besar. Luigi juga memungkinkan kamu untuk membangun pipeline yang sederhana dengan cepat. Ini membantu bundling library yang dibutuhkan, serta mengembalikan error log ke komputer. Spotify juga mengaplikasikan Luigi bersama dengan berbagai algoritma machine learning untuk menghidupkan fitur Radio dan Discover, serta rekomendasi untuk orang yang mungkin ingin kamu ikuti. Di halaman Spotify Labs, Spotify menggunakan bahasa pemrograman Python dalam sembilan puluh persen urusan MapReduce mereka. Data Science Graduate Program mengibaratkan Hadoop sebagai sumber hidup big data, sementara MapReduce berperan sebagai detak jantungnya}

\end{document}
